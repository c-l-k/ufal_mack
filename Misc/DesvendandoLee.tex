\documentclass[12pt]{article}

\usepackage{fourier}
\usepackage[portuguese]{babel}
\usepackage[utf8]{inputenc}
\usepackage{natbib}
\usepackage{bm}
\usepackage{amsmath}
\newcommand{\E}{\operatorname{E}}

\title{Desvendando a Notação de Lee}
\author{}
\date{}

\begin{document}
\maketitle

\section{O modelo}
Reescreveremos a notação que \citet{lee} utilizam, pois ela nem sempre é clara ou precisa.

O vetor de retornos complexos é
$$
\bm S =
	\begin{bmatrix}
	S_{\text{HH}}\\
	S_{\text{HV}}\\
	S_{\text{VV}}
	\end{bmatrix},
$$
e a sua distribuição é gaussiana complexa de média nula e matriz de covariância $\Sigma=\E\big(\bm S \bm S^{*T}\big)$.
Com isso,
$$
\Sigma= \E\begin{pmatrix}
\|S_{\text{HH}}\|	& S_{\text{HH}} S_{\text{HV}}^{*T}	& S_{\text{HH}} S_{\text{VV}}^{*T}\\
				& \|S_{\text{HV}}\|					& S_{\text{HV}} S_{\text{VV}}^{*T}\\
				&										& \|S_{\text{VV}}\|
\end{pmatrix},
$$
em que omitimos as entradas que são os conjungados das simétricas.
Os elementos da diagonal de $\Sigma$ são as esperanças das intensidades,
enquanto os elementos fora da diagonal são as covarâncias complexas entre os canais.
Podemos, então usar a seguinte notação:
$$
\Sigma = \begin{pmatrix}
\sigma_{\text{HH}}^2		& \sigma_{\text{HH,HV}}		& \sigma_{\text{HH,VV}} \\
\sigma_{\text{HH,HV}}^* 	& \sigma_{\text{HV}}^2		& \sigma_{\text{HV,VV}} \\
\sigma_{\text{HH,VV}}^*		& \sigma_{\text{HV,VV}}^*	& \sigma_{\text{VV}}^2
\end{pmatrix}.
$$
Reforçamos que se trata de parâmetros e, portanto, de quantidades fixas porém, em geral, desconhecidas.

Estamos interessados em obter distribuições de transformações destes dados, dentre elas a distribuição da matriz de covariância amostral multilook (repare no uso que eu faço da palavra \textit{amostral} para evitar a confusão da matriz aleatória $\bm Z$ com o parâmetro $\Sigma$):
$$
\bm Z = \frac1L \sum_{\ell=1}^L \bm S(\ell) S^{*T}(\ell) = 
\begin{pmatrix}
I_{\text{HH}}	& C_{\text{HH,HV}}	& C_{\text{HH,VV}} \\
				& I_{\text{HV}}		& C_{\text{HV,VV}} \\
				&					& I_{\text{VV}} 
\end{pmatrix},
$$
que segue uma lei Wishart com parâmetros $L$ e $\Sigma$.

Uma quantidade de interesse é a razão das intensidades, por exemplo $R_{\text{HH,VV}} = I_{\text{HH}} / I_{\text{VV}}$.
\citet{lee} tratam de forma genérica a razão de intensidades normalizada que, para o nosso caso, é
$$
R_{\text{HH,VV}} = \frac{\frac1{\sigma_{\text{HH}}^2}I_{\text{HH}}}
						{\frac1{\sigma_{\text{VV}}^2}I_{\text{VV}}} = 
	\frac{\frac1{\sigma_{\text{HH}}^2} \frac1L \sum_{\ell=1}^{L}\|S_{\text{HH}}(\ell)\|}
	{\frac1{\sigma_{\text{VV}}^2} \frac1L \sum_{\ell=1}^{L}\|S_{\text{VV}}(\ell)\|}.
$$
Denotando $\tau = \sigma_{\text{HH}}^2 / \sigma_{\text{VV}}^2$ e simplificando, temos
$$
R_{\text{HH,VV}} = \frac{1}{\tau}
\frac{\sum_{\ell=1}^{L}\|S_{\text{HH}}(\ell)\|}
{\sum_{\ell=1}^{L}\|S_{\text{VV}}(\ell)\|}.
$$
Verificando os parâmetros de cada componente de $\bm Z$, concluímos que $R_{\text{HH,VV}}$ depende, no máximo, de $L$, $\sigma_{\text{HH}}^2$, $\sigma_{\text{VV}}^2$, e de $C_{\text{HH,HV}}$.
Note-se que $\tau$ é a razão de dois parâmetros e, portanto, não é um novo parâmetro.

\citet{lee} obtêm a distribuição de $R_{\text{HH,VV}}$ de tal forma que não depende da covariância complexa $C_{\text{HH,HV}}$, mas sim do seu módulo $\rho_{\text{HH,VV}}=|C_{\text{HH,HV}}|$, e que depende apenas da razão das intensidades médias $\tau$.
Com isso, esta distribuição envolve três parâmetros.


\section{Método da verossimilhança aplicado na PDF univariada razão de intensidades múltiplas visadas}
A razão de intensidades ou amplitudes entre $\mathbf{S}_i$ e $\mathbf{S}_j$ são importantes no estudo de radares polarimétricos. Seja a razão de intensidade normalizada,
\begin{equation}\label{eq:razao_intensidades}
 \mu=R_{\text{HH,VV}} = \frac{1}{\tau}
\frac{\sum_{\ell=1}^{L}\|S_{\text{HH}}(\ell)\|}
{\sum_{\ell=1}^{L}\|S_{\text{VV}}(\ell)\|}.
\end{equation}
onde $\tau=\frac{\sigma^2_{\text{HH}}}{\sigma^2_{\text{VV}}}$.
  
Considerando a função densidade de probabilidade univariada razão de intensidades múltiplas visadas,
\begin{equation}\label{eq:pdf_razao_intensidades}
	f(\mu;\rho,L)=\frac{\Gamma(2L)(1-|\rho|^2)^{L}(1+\mu)\mu^{L-1}}{\Gamma(L)\Gamma(L)\left[(1+\mu)^2-4|\rho|^2\mu \right]^{\frac{2L+1}{2}}}\\
\end{equation}
onde, $\rho>0$ e $L>0$. Aplicando o logaritmo natural na equação~\eqref{eq:pdf_razao_intensidades} e realizando algumas manipulações algébricas teremos:
\begin{equation}\nonumber
\begin{split}
	\ln f(\mu;\rho,L)&=\ln\left(\frac{\Gamma(2L)(1-|\rho|^2)^{L}(1+\mu)\mu^{L-1}}{\Gamma(L)\Gamma(L)\left[(1+\mu)^2-4|\rho|^2\mu \right]^{\frac{2L+1}{2}}}\right),\\
	                &=\ln\left(\Gamma(2L)(1-|\rho|^2)^{L}(1+\mu)\mu^{L-1}\right)\\
	                &-\ln\left(\Gamma(L)\Gamma(L)\left[(1+\mu)^2-4|\rho|^2\mu \right]^{\frac{2L+1}{2}}\right),\\
	                &=\ln\Gamma(2L) +\ln(1-|\rho|^2)^{L}+\ln(1+\mu)+\ln\mu^{L-1}\\
	                &-\left(\ln\Gamma(L)+\ln\Gamma(L)+\ln\left[(1+\mu)^2-4|\rho|^2\mu \right]^{\frac{2L+1}{2}}\right),\\
	                &=\ln\Gamma(2L) +L\ln(1-|\rho|^2)+\ln(1+\mu)+(L-1)\ln\mu\\
	                &-2\ln\Gamma(L)-\frac{2L+1}{2}\ln\left[(1+\mu)^2-4|\rho|^2\mu \right].\\
\end{split}
\end{equation}

Definimos a equação log-verossimilhança para a PDF univariada razão de intensidades múltiplas visadas,
\begin{equation}\label{eq_log_vero_razao_intensidade}
\begin{array}{lcl}	
	\ln f(\mu;\rho,L)&=&\ln\Gamma(2L) +L\ln(1-|\rho|^2)+\ln(1+\mu)+(L-1)\ln\mu\\
	&-&2\ln\Gamma(L)-\frac{2L+1}{2}\ln\left[(1+\mu)^2-4|\rho|^2\mu \right].\\
\end{array}
\end{equation}

A função log-verossimilhança pode ser deduzida da seguinte maneira, dado a amostra $\bm\mu = (\mu_1,\dots,\mu_n)$, 
\begin{equation}\nonumber
\begin{split}
  \ell(\bm \mu;\rho, L)=\ln\prod_{k=1}^{n}f(\mu_k;\rho,L)\\
  \ell(\bm \mu;\rho, L)=\sum_{k=1}^{n}\ln f(\mu_k;\rho,L),
 \end{split}
 \end{equation}
usando a função~\eqref{eq_log_vero_razao_intensidade} teremos,
\begin{equation}\nonumber
\begin{split}
    \ell(\bm \mu;\rho, L)&=\sum_{k=1}^{n}\ln f(\mu_k;\rho, L)\\
                         &=\sum_{k=1}^{n}\left[\ln\Gamma(2L) +L\ln(1-|\rho|^2)+\ln(1+\mu_k)+(L-1)\ln\mu_k\right.\\
                         &-\left.2\ln\Gamma(L)-\frac{2L+1}{2}\ln\left[(1+\mu_k)^2-4|\rho|^2\mu_k\right]\right]\\
 \end{split}
 \end{equation}
 
 \begin{equation}\nonumber
\begin{split} 
    \ell(\bm \mu;\rho, L)&=\ln\Gamma(2L)\sum_{k=1}^{n} 1+L\ln(1-|\rho|^2)\sum_{k=1}^{n} 1+\sum_{k=1}^{n}\ln(1+\mu_k)+(L-1)\sum_{k=1}^{n}\ln\mu_k\\
                         &-2\ln\Gamma(L)\sum_{k=1}^{n} 1-\frac{2L+1}{2}\sum_{k=1}^{n}\ln\left[(1+\mu_k)^2-4|\rho|^2\mu_k\right]\\
                         &=n\left(\ln\Gamma(2L)+L\ln(1-|\rho|^2)-2\ln\Gamma(L)\right)+\sum_{k=1}^{n}\ln(1+\mu_k)\\
                         &+L\sum_{k=1}^{n}\ln\mu_k-\sum_{k=1}^{n}\ln\mu_k-\frac{2L+1}{2}\sum_{k=1}^{n} \ln\left[(1+\mu_k)^2-4|\rho|^2\mu_k\right]\\
\end{split}
\end{equation}
 
Definimos a equação log-verossimilhança para a PDF univariada~(\ref{eq_log_vero_razao_intensidade}).
\begin{equation}\nonumber
\begin{split}
    \ell(\bm \mu;\rho, L)&=n\left(\ln\Gamma(2L)+L\ln(1-|\rho|^2)-2\ln\Gamma(L)\right)+\sum_{k=1}^{n}\ln(1+\mu_k)\\
                         &+L\sum_{k=1}^{n}\ln\mu_k-\sum_{k=1}^{n}\ln\mu_k-\frac{2L+1}{2}\sum_{k=1}^{n} \ln\left[(1+\mu_k)^2-4|\rho|^2\mu_k\right]\\
\end{split}
 \end{equation}
e a forma reduzida,
\begin{equation}\label{eq_log_vero_razao_intensidade_red}
\begin{split}
    \ell(\bm \mu;\rho, L)&=n\left(\ln\Gamma(2L)+L\ln(1-|\rho|^2)-2\ln\Gamma(L)\right)\\
                         &+L\sum_{k=1}^{n}\ln\mu_k-\frac{2L+1}{2}\sum_{k=1}^{n} \ln\left[(1+\mu_k)^2-4|\rho|^2\mu_k\right]\\
\end{split}
 \end{equation} 

Vamos obter $(\widehat \rho, \widehat L)$, o estimador de máxima verossimilhança (MLE) de $(\rho, L)$ baseado $\bm \mu$, por maximizar~\eqref{eq_log_vero_razao_intensidade_red} com o método BFGS implementado no pacote \texttt{maxLik}~\citep{ht}. Vamos preferir otimização resolvendo $\nabla\ell=\bm 0$ com intuito de melhorar a estabilidade numérica.

O função é a log-verossimilhança reduzida para as amostras internas e externas da faixa de dados denotadas respectivamento como $\bm \mu_\text{I}$ e $\bm \mu_\text{E}$. Cada faixa de dados $\bm \mu = (\mu_1,\mu_2,\dots,\mu_n)$ é particionada em duas amostras disjuntas na posição $j$:  
$$
\bm \mu = (\underbrace{\mu_1,\mu_2,\dots,\mu_j}_{\bm \mu_\text{I}}, 
\underbrace{\mu_{j+1}, \mu_{j+2},\dots,\mu_n}_{\bm \mu_\text{E}}).
$$
%Vamos assumir dois diferentes modelos para cada partição:
%$\bm Z_\text{I} \sim \Gamma(\mu_\text{I},L_\text{I})$, e
%$\bm Z_\text{E} \sim \Gamma(\mu_\text{E},L_\text{E})$.
Vamos estimar $(\rho_\text{I},L_\text{I})$ e $(\rho_\text{E},L_\text{E})$ com $\bm \mu_\text{I}$ e $\bm \mu_\text{E}$, respectivamente, maximizando~\eqref{eq:eq_log_vero_mag_prod_red}, e obtendo $(\widehat{\rho}_\text{I}, \widehat{L}_\text{I})$ e $(\widehat{\rho}_\text{E}, \widehat{L}_\text{E})$.

A log-verossimilhança no ponto $j$ é, então
\begin{equation}\label{eq:TotalLogLikelihood}
\begin{split}
\ell(j;\widehat{\rho}_\text{I}, \widehat{L}_\text{I}, \widehat{\rho}_\text{E}, \widehat{L}_\text{E})&=n\left(\ln\Gamma(2\widehat{L}_\text{I})+\widehat{L}_\text{I}\ln(1-|\widehat{\rho}_\text{I}|^2)-2\ln\Gamma(\widehat{L}_\text{I})\right)\\
                         &+\widehat{L}_\text{I}\sum_{k=1}^{n}\ln\mu_k-\frac{2\widehat{L}_\text{I}+1}{2}\sum_{k=1}^{n} \ln\left[(1+\mu_k)^2-4|\widehat{\rho}_\text{I}|^2\mu_k\right]\\
                         &=n\left((\ln\Gamma(2\widehat{L}_\text{E})+\widehat{L}_\text{E}\ln(1-|\widehat{\rho}_\text{E}|^2)-2\ln\Gamma(\widehat{L}_\text{E})\right)\\
                         &+\widehat{L}_\text{E}\sum_{k=1}^{n}\ln\mu_k-\frac{2\widehat{L}_\text{E}+1}{2}\sum_{k=1}^{n} \ln\left[(1+\mu_k)^2-4|\widehat{\rho}_\text{E}|^2\mu_k\right]
%\raisetag{2.2em}
\end{split}
\end{equation}

Vamos aplicar o método GenSA para encontrar
$$
\widehat{\jmath}= \arg\max\limits_{j\in [\min_s,N-\min_s]}\ell(j;\widehat{\rho}_I, \widehat{L}_I,\widehat{\rho}_E, \widehat{L}_E),
$$ 
onde $\min_s$ é o tamanho mínimo da amostra definido por $14$.

Desta maneira, vamos obter uma estimativa para a borda em cada canal de intensidade.
Note que esse método pode ser estendido e/ou modificado para lidar com qualquer tipo de dados.


\bibliographystyle{agsm}
\bibliography{../Text/bibliografia}

\end{document}