\documentclass[12pt]{article}

\usepackage{fourier}
\usepackage[portuguese]{babel}
\usepackage[utf8]{inputenc}
\usepackage{natbib}
\usepackage{bm}
\usepackage{amsmath}
\newcommand{\E}{\operatorname{E}}

\title{Desvendando a Notação de Lee}
\author{}
\date{}

\begin{document}
\maketitle

Reescreveremos a notação que \citet{lee} utilizam, pois ela nem sempre é clara ou precisa.

O vetor de retornos complexos é
$$
\bm S =
	\begin{bmatrix}
	S_{\text{HH}}\\
	S_{\text{HV}}\\
	S_{\text{VV}}
	\end{bmatrix},
$$
e a sua distribuição é gaussiana complexa de média nula e matriz de covariância $\Sigma=\E\big(\bm S \bm S^{*T}\big)$.
Com isso,
$$
\Sigma= \E\begin{pmatrix}
\|S_{\text{HH}}\|^2	& S_{\text{HH}} S_{\text{HV}}^{*T}	& S_{\text{HH}} S_{\text{VV}}^{*T}\\
				& \|S_{\text{HV}}\|^2					& S_{\text{HV}} S_{\text{VV}}^{*T}\\
				&										& \|S_{\text{VV}}\|^2
\end{pmatrix},
$$
em que omitimos as entradas que são os conjungados das simétricas.
Os elementos da diagonal de $\Sigma$ são as esperanças das intensidades,
enquanto os elementos fora da diagonal são as covarâncias complexas entre os canais.
Podemos, então usar a seguinte notação:
$$
\Sigma = \begin{pmatrix}
\sigma_{\text{HH}}^2		& \sigma_{\text{HH,HV}}		& \sigma_{\text{HH,VV}} \\
\sigma_{\text{HH,HV}}^* 	& \sigma_{\text{HV}}^2		& \sigma_{\text{HV,VV}} \\
\sigma_{\text{HH,VV}}^*		& \sigma_{\text{HV,VV}}^*	& \sigma_{\text{VV}}^2
\end{pmatrix}.
$$
Reforçamos que se trata de parâmetros e, portanto, de quantidades fixas porém, em geral, desconhecidas.

Estamos interessados em obter distribuições de transformações destes dados, dentre elas a distribuição da matriz de covariância amostral multilook (repare no uso que eu faço da palavra \textit{amostral} para evitar a confusão da matriz aleatória $\bm Z$ com o parâmetro $\Sigma$):
$$
\bm Z = \frac1L \sum_{\ell=1}^L \bm S(\ell) S^{*T}(\ell) = 
\begin{pmatrix}
I_{\text{HH}}	& C_{\text{HH,HV}}	& C_{\text{HH,VV}} \\
				& I_{\text{HV}}		& C_{\text{HV,VV}} \\
				&					& I_{\text{VV}} 
\end{pmatrix},
$$
que segue uma lei Wishart com parâmetros $L$ e $\Sigma$.

Uma quantidade de interesse é a razão das intensidades, por exemplo $R_{\text{HH,VV}} = I_{\text{HH}} / I_{\text{VV}}$.
\citet{lee} tratam de forma genérica a razão de intensidades normalizada que, para o nosso caso, é
$$
R_{\text{HH,VV}} = \frac{\frac1{\sigma_{\text{HH}}^2}I_{\text{HH}}}
						{\frac1{\sigma_{\text{VV}}^2}I_{\text{VV}}} = 
	\frac{\frac1{\sigma_{\text{HH}}^2} \frac1L \sum_{\ell=1}^{L}\|S_{\text{HH}}(\ell)\|^2}
	{\frac1{\sigma_{\text{VV}}^2} \frac1L \sum_{\ell=1}^{L}\|S_{\text{VV}}(\ell)\|^2}.
$$
Denotando $\tau = \sigma_{\text{HH}}^2 / \sigma_{\text{VV}}^2$ e simplificando, temos
$$
R_{\text{HH,VV}} = \frac{1}{\tau}
\frac{\sum_{\ell=1}^{L}\|S_{\text{HH}}(\ell)\|^2}
{\sum_{\ell=1}^{L}\|S_{\text{VV}}(\ell)\|^2}.
$$
Verificando os parâmetros de cada componente de $\bm Z$, concluímos que $R_{\text{HH,VV}}$ depende, no máximo, de $L$, $\sigma_{\text{HH}}^2$, $\sigma_{\text{VV}}^2$, e de $C_{\text{HH,HV}}$.
Note-se que $\tau$ é a razão de dois parâmetros e, portanto, não é um novo parâmetro.



\bibliographystyle{agsm}
\bibliography{../Text/bibliografia}

\end{document}