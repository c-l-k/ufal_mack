\documentclass[conference]{IEEEtran}
\IEEEoverridecommandlockouts
% The preceding line is only needed to identify funding in the first footnote. If that is unneeded, please comment it out.
\usepackage{cite}
\usepackage{amsmath,amssymb,amsfonts}
\usepackage{algorithmic}
\usepackage{graphicx}
\usepackage{textcomp}
\usepackage{xcolor}
\usepackage{booktabs}                   % AAB inserido
\usepackage[utf8]{inputenc}             % AAB inserido
\usepackage{rotating}                   % AAB inserido
\def\BibTeX{{\rm B\kern-.05em{\sc i\kern-.025em b}\kern-.08em
    T\kern-.1667em\lower.7ex\hbox{E}\kern-.125emX}}
%AAB
\DeclareMathOperator{\traco}{tr}
\begin{document}

\title{Paper Title*\\
{\footnotesize \textsuperscript{*}Note: Sub-titles are not captured in Xplore and
should not be used}
\thanks{Bolsista Capes.}
}

\author{\IEEEauthorblockN{Anderson A. de Borba}
\IEEEauthorblockA{\textit{Dept. Engenharia Elétrica e Computação} \\
\textit{UPM - Universidade Presbiteriana Mackenzie}\\
IBMEC\\
São Paulo, Brazil \\
anderson.borba@ibmec.edu.br}
\and
\IEEEauthorblockN{Maurício Marengoni}
\IEEEauthorblockA{\textit{Dept. Engenharia Elétrica e Computação} \\
\textit{UPM - Universidade Presbiteriana Mackenzie}\\
São Paulo, Brazil \\
mauricio.marengoni@mackenzie.br}
\and
\IEEEauthorblockN{\hspace{6cm} Alejandro C. Frery}
\IEEEauthorblockA{\textit{\hspace{6cm}Laboratório de Computação Científica e Análise Numérica} \\
\hspace{6cm}\textit{UFAL - Universidade Federal de Alagoas)}\\
\hspace{6cm} Maceió, Brazil \\
\hspace{6cm}acfrery@gmail.com}
}

\maketitle

\begin{abstract}
This document is a model and instructions for \LaTeX.
This and the IEEEtran.cls file define the components of your paper [title, text, heads, etc.]. *CRITICAL: Do Not Use Symbols, Special Characters, Footnotes, 
or Math in Paper Title or Abstract.
\end{abstract}

\begin{IEEEkeywords}
component, formatting, style, styling, insert
\end{IEEEkeywords}

\section{Introduction}
\section{Modelagem estatística para dados PolSAR}
Os sistemas SAR totalmente polarimétricos transmitem pulsos de micro-ondas polarizados ortogonalmente e medem componentes ortogonais do sinal recebido. Para cada pixel, a medida resulta em uma matriz de coeficientes de espalhamento. Esses coeficientes são números complexos que descrevem no sistema SAR a transformação do campo eletromagnético transmitido para o campo eletromagnético recebido.

A transformação pode ser representada como
\begin{equation}
 \left[
\begin{array}{c}
	E_{h}^{r}   \\
	E_{v}^{r}    \\
\end{array}
\right]
 = \frac{e^{\hat{\imath} kr}}{r}\left[
\begin{array}{cc}
	S_{hh}   & S_{hv}   \\
	S_{vh}   & S_{vv}   \\
\end{array}
\right]
 \left[
\begin{array}{c}
	E_{h}^{t}   \\
	E_{v}^{t}    \\
\end{array}
\right],
\end{equation}
onde $k$ denota o número de onda, $\hat{\imath}$ é um número complexo e $r$ é a distância entre o radar e o alvo. O campo eletromagnético com componentes $E_{i}^{j}$, o índice subscrito denota polarização horizontal ($h$) ou vertical ($v$),  enquanto o índice sobrescrito indica a onda recebida ($r$) ou transmitida ($t$). Definindo $S_{i,j}$ como os coeficientes de espalhamento complexo, tal que o índice $i$ e $j$ são associados com o recebimento com a transmissão das ondas, por exemplo, o coeficiente de espalhamento $S_{hv}$ está associado a onda transmitida na direção vertical ($v$) e recebida na direção horizontal ($h$).

Sendo conhecido cada um dos coeficientes, a matriz de espalhamento complexa $\mathbf{S}$ é definida por
\begin{equation}
\mathbf{S} = \left[
\begin{array}{cc}
	S_{hh}   & S_{hv}   \\
	S_{vh}   & S_{vv}   \\
\end{array}
\right],
\end{equation}
se o meio de propagação das ondas é recíproco, isto é, de uma maneira geral as propriedades de transmissão e recebimento de uma antena são idênticos, então usaremos o teorema da reciprocidade \cite{lp} para definir a matriz de espalhamento como sendo hermitiana, ou seja, a igualdade dos termos complexos $S_{hv}=S_{vh}$. De acordo com o teorema da reciprocidade a matriz de espalhamento pode ser representada pelo vetor
\begin{equation}
\mathbf{S} = \left[
\begin{array}{c}
	S_{hh}     \\
    S_{vh}     \\
	S_{vv}     \\
\end{array}
\right].
\end{equation}

Sabemos que cada componente do vetor $\mathbf{S}$ é complexo, com intuito de fixar notação reescrevemos,
\begin{equation}
\begin{array}{ccc}
	S_{hh} && R_{hh} + i I_{hh}    \\
    S_{vh} &=& R_{hv} + i I_{hv}   \\
	S_{vv} &=& R_{vv} + i I_{vv}   \\
\end{array}
\end{equation}
Poderíamos considerar um vetor de dimensão $6$ onde cada entrada está distribuída como $N(0,\sigma)$.
\begin{equation}
\mathbf{S} = \left[
\begin{array}{c}
	R_{hh}     \\
    I_{hh}     \\
	R_{hv}     \\
	I_{hh}     \\
    R_{vv}     \\
	I_{vv}     \\
\end{array}
\right].
\end{equation}

Por hipótese teremso a distribuição gaussiana circular complexa multivariada com média zero pode ser definida de acordo com \cite{good}, assim, sendo $\mathbf{S}_{ij}= R_{ij}+ i I_{ij}$ é exigido que $R_{ij}$ e $y_{ij}$ com $j=h,v$ tenham distribuições conjuntas gaussianas e satisfaçam as seguintes condições 
\begin{itemize}
    \item[-] obs: entender e reescrever melhor essas hipóteses,
	\item[-] $E[R_{ij}]=E[I_{ij}]=0$,
	\item[-] $E[R_j^2]=E[I_j^2]$,
	\item[-] $E[R_jI_j]=0$,
	\item[-] $E[R_jR_i]=E[I_jI_i]$,
	\item[-] $E[I_jR_i]=-E[R_jR_i]$,
\end{itemize}
onde, $E[\cdot]$ é o valor esperado.


De acordo com \cite{good} e \cite{lee} esta distribuição pode modelar adequadamente o comportamento estatístico de $\mathbf{S}$. A hipotêse de ser gaussiana e circular foi comprovada para dados SAR polarimétricos no artigo \cite{sarabendi}.   


A matriz de covariância associada a $\mathbf{S}$ definida por
\begin{equation}
	{\bf C_{{\bf S}}} = E[{\mathbf S}{\mathbf S}^H] = \left[
\begin{array}{cc}
	E[S_{hh}\overline{S_{hh}}]  & E[{S_{hh}}\overline{S_{hv}}]   \\
	E[S_{hv}\overline{S_{hh}}]  & E[{S_{vv}}\overline{S_{vv}}]  \\
\end{array}
\right]
\end{equation}
talque, $\overline{\cdot}$ denota o conjugado complexo. A matriz de covariância é hermitiana positiva definida e contém todas as informações necessárias para caracterizar o retroespalhamento, podemos consultar mais informações em \cite{mfp}. 

Nas imagens PolSAR serão consideradas três componentes para o vetor $\mathbf{S}=[S_{hh},S_{vh},S_{vv}]^T$ e a multiplicação de $\mathbf{s}=[S_{hh},S_{vh},S_{vv}]$ pelo seu conjugado transposto $\mathbf{S}=[S_{hh},S_{vh},S_{vv}]^H$, isto é, a hermitiana do vetor, 

\begin{equation}
\mathbf{s}\mathbf{s}^H = \left[
\begin{array}{c}
	S_{hh}      \\
        S_{vh}     \\
	S_{vv}      \\
\end{array}
\right]
\left[
\begin{array}{ccc}
	S_{hh}  & S_{vh}  & S_{vv}      \\
\end{array}
\right]^H = \left[S_{ij} \right]_{i,j=h,v}
\end{equation}

De acordo com \cite{good} a distribuição gaussiana complexa multivariada pode modelar adequadamente o comportamento estatístico de $\boldmath S$. Isto é chamado de {\it single-look complex PolSAR data representation} e podemos definir o vetor de espalhamento por $\mathbf{S}=[S_{hh},S_{hv},S_{vv}]^H$. 

Dados polarimétricos são usualmente sujeitados a um processo de várias visadas com o intuito de melhorar a razão entre o sinal e o seu ruído. Para esse fim, matrizes positivas definidas hermitianas estimadas são obtidas computando a média de $L$ visadas independentes de uma mesma cena. Resultando na matriz de covariância amostral estimada {\bf Z} conforme \cite{good, ade}
\begin{equation}
\begin{array}{ccc}
    \mathbf{Z}&=&\frac{1}{L}\displaystyle{\sum_{l=1}^{L} {\mathbf{s}_l}{\mathbf{s}_l}^H}, \\
\end{array}
\end{equation}
onde $\mathbf{s}_l$ com $l = 1, \dots, L$ é uma amostra de $\mathit{L}$ vetores complexos distribuídos como $\mathbf{S}$, assim a matriz de covariância amostral associada a $\mathbf{S}_l$, com $l=1,\dots,L$ denotam o espalhamento para cada visada $L$

Sendo $i=j$
\begin{equation}
\begin{array}{ccc}
\mathbf{S}_{ii}\overline{\mathbf{S}}_{ii}&=& (R_{ii}+iI_{ii})\overline{(R_{ii}+iI_{jj})} \\
\mathbf{S}_{ii}\overline{\mathbf{S}}_{ii}&=& (R_{ii}+iI_{ii})(R_{ii}-iI_{ii}) \\
\mathbf{S}_{ii}\overline{\mathbf{S}}_{ii}&=& R_{ii}^2+I_{ii}^2 \\
\end{array}
\end{equation}
e considerando $i \neq j$
\begin{equation}
\begin{array}{ccc}
\mathbf{S}_{ii}\overline{\mathbf{S}}_{ij}&=& (R_{ii}+iI_{ii})\overline{(R_{ij}+iI_{ij})} \\
\mathbf{S}_{ii}\overline{\mathbf{S}}_{ij}&=& (R_{ii}+iy_{ii})(I_{ij}-iI_{ij}) \\
\mathbf{S}_{ii}\overline{\mathbf{S}}_{ij}&=& (R_{ii}R_{ij}+I_{ii}I_{ij})+i(R_{ij}I_{ii}-R_{ii}I_{ij}) \\
\end{array}
\end{equation}
 Definindo,
 \begin{equation}
\begin{array}{ccc}
	  RC_{ij}&=&  R_{ii}R_{ij}+I_{ii}I_{ij} 
\end{array}
\end{equation}
e
\begin{equation}
\begin{array}{ccc}
	  IC_{ij}&=& R_{ij}I_{ii}-R_{ij}I_{ii}
\end{array}
\end{equation}

Sendo a variável randômica gaussiana complexa $\mathbf{C_{i,j}}=RC_{ij} + i IC_{ij}$, ou ainda, $\mathbf{C_{i,j}}=(R_{ii}R_{ij}+I_{ii}I_{ij}) + i(R_{ij}I_{ii}-R_{ij}I_{ii})$. Podemos escrever uma variável randômica gaussiana complexa $4-$variada $(R_{ii},R_{ij},I{ii},I_{ij})$.

De acordo com (\cite{good})
\begin{equation}
\mathbf{C} = \left[
\begin{array}{cc}
	E(X_iX_j)  & E(X_iY_j)  \\
	E(Y_iX_j)  & E(Y_iY_j)  \\
\end{array}
\right].
\end{equation}
Tal que
\begin{equation}
\mathbf{C} =
\left\{
\begin{array}{cc}
	\frac{1}{2}\left[
\begin{array}{cc}
	 1 & 0  \\
	 0 & 1  \\
\end{array}
	\right]\sigma^{2}_{j}  & \mbox{se}\quad i=j, \\
	& \\
	\frac{1}{2}\left[
\begin{array}{cc}
	\alpha_{ij} & -\beta_{ij}  \\
	 \beta_{ij} & \alpha_{ij}  \\
\end{array}
	\right]\sigma_j\sigma_k  & \mbox{se}\quad i\neq j.   \\
\end{array}
\right.
\end{equation}

\begin{sidewaystable}
	\centering
	\caption{Tabela}
\begin{tabular}{@{}lcccccc@{}} \toprule
	     &$R_{hh}$        & $I_{hh}$ & $R_{hv}$&$I_{hv}$                            &$R_{vv}$                           &$I_{vv}$ \\ \midrule
$R_{hh}$ &$\sigma_{hh}^2$ & 0                  &$\rho_{hh,hv}\sigma_{hh}\sigma_{hv}$ &$\eta_{hh,hv}\sigma_{hh}\sigma_{hv}$ & $\rho_{hh,vv}\sigma_{hh}\sigma_{vv}$&$\eta_{hh,vv}\sigma_{hh}\sigma_{vv}$  \\ 
	$I_{hh}$ & 0 & $\sigma_{hh}^2$ &$-\eta_{hh,hv}\sigma_{hh}\sigma_{hv}$ &$\rho_{hh,hv}\sigma_{hh}\sigma_{hv}$ &$-\eta_{hh,vv}\sigma_{hh}\sigma_{vv}$ &$\rho_{hh,vv}\sigma_{hh}\sigma_{vv}$  \\ 
	$R_{hv}$ &$\rho_{hh,hv}\sigma_{hh}\sigma_{hv}$   &$-\eta_{hh,hv}\sigma_{hh}\sigma_{hv}$  &$\sigma_{hv}^2$ &0 &$\rho_{hv,vv}\sigma_{hv}\sigma_{vv}$ &$\eta_{hv,vv}\sigma_{hv}\sigma_{vv}$  \\ 
	$I_{hv}$ &$\eta_{hh,hv}\sigma_{hh}\sigma_{hv}$  &$\rho_{hh,hv}\sigma_{hh}\sigma_{hv}$  &0 &$\sigma_{hv}^2$ &$-\eta_{hv,vv}\sigma_{hv}\sigma_{vv}$ &$\rho_{hv,vv}\sigma_{hv}\sigma_{vv}$ \\ 
	$R_{vv}$ &$\rho_{hh,vv}\sigma_{hh}\sigma_{vv}$  &$-\eta_{hh,vv}\sigma_{hh}\sigma_{vv}$  &$\rho_{hv,vv}\sigma_{hv}\sigma_{vv}$ &$-\eta_{hv,vv}\sigma_{hv}\sigma_{vv}$ & $\sigma_{vv}^2$ &0 \\ 
    $I_{vv}$ &$\eta_{hh,vv}\sigma_{hh}\sigma_{hv}$  &$\rho_{hh,vv}\sigma_{hh}\sigma_{vv}$  &$\eta_{hv,vv}\sigma_{hv}\sigma_{vv}$ &$\rho_{hv,vv}\sigma_{hv}\sigma_{vv}$ & 0 &$\sigma_{vv}^2$ \\ 	 \bottomrule
\end{tabular}
\end{sidewaystable}
\section{Funções de densidade}
Sendo $(R_{ii}, R_{ij})\sim N2(0, C_{ij})$ podemos observar na tabela anterior que 
\begin{equation}
C_{ij}=\left[
\begin{array}{cc}
	\sigma_{ii}^2   &  \rho_{ii,ij}\sigma_{ii}\sigma_{ij}  \\
	\rho_{ii,ij}\sigma_{ii}\sigma_{ij} & \sigma_{ij}^2   \\
\end{array}
\right],
\end{equation}
A pdf para esta distribuição normal é:

\begin{equation}
\begin{array}{ccc}
	f_{Z_{R_{ii}R_{ij}}}(z)&=&\frac{1}{\pi\sigma_{ii}\sigma_{ij}\sqrt{1-\rho_{ii,ij}^2}}\exp\left(\frac{\rho_{ii,ij}z}{\sigma_{ii}\sigma_{ij}(1-\rho_{ii,ij})^2}\right)\\
	&&K_0\left(\frac{|z|}{(\sigma_{ii}\sigma_{ij}(1-\rho_{ii,ij})^2}\right).
\end{array}
\end{equation}

Definindo o funcional $\Theta(z;\sigma_p,\sigma_q,\gamma)$
\begin{equation}
\begin{array}{ccc}
	\Theta(z;\sigma_p,\sigma_q,\gamma)&=&\frac{1}{\pi\sigma_p\sigma_q\sqrt{1-\gamma^2}}\exp\left(\frac{\gamma z}{\sigma_p\sigma_q(1-\gamma)^2}\right)\\
	&&K_0\left(\frac{|z|}{(\sigma_p\sigma_q(1-\gamma)^2}\right).
\end{array}
\end{equation}
Sendo $(I_{ii}, I_{ij})\sim N2(0, C_{ij})$ podemos observar na tabela anterior que 
\begin{equation}
C_{ij}=\left[
\begin{array}{cc}
	\sigma_{ii}^2   &  \rho_{ii,ij}\sigma_{ii}\sigma_{ij}  \\
	\rho_{ii,ij}\sigma_{ii}\sigma_{ij} & \sigma_{ij}^2   \\
\end{array}
\right],
\end{equation}
obs: mesma distribuição!!!!!!

\section{distribuicão conjunta}
Sendo $(R_{ii}, R_{ij},I_{ii}, I_{ij})\sim N2(0, C_{ii,ij})$ podemos observar na tabela anterior que 
\begin{equation}
C_{ii,ij}=\left[
\begin{array}{cccc}
	\sigma_{ii}^2   &  \rho_{ii,ij}\sigma_{ii}\sigma_{ij} & 0&\eta_{ii,ij}\sigma_{ii}\sigma_{ij}\\
	\rho_{ii,ij}\sigma_{ii}\sigma_{ij} & \sigma_{ij}^2  & -\eta_{ii,ij}\sigma_{ii}\sigma_{ij}&0 \\
	0&-\eta_{ii,ij}\sigma_{ii}\sigma_{ij}&\sigma_{ii}^2&\rho_{ii,ij}\sigma_{ii}\sigma_{ij}\\
	\eta_{ii,ij}\sigma_{ii}\sigma_{ij}&0&\rho_{ii,ij}\sigma_{ii}\sigma_{ij}&\sigma_{ij}^2\\
\end{array}
\right],
\end{equation}
Realizar a transformação 
\begin{equation}
\left[
\begin{array}{ccc}
	 Z = R_{ii}R_{ij}+I_{ii}I_{ij} \\
	 U_1 = R_{ii}\\
	 U_2 = R_{ij}\\
	 U_3 = I_{ii}\\
\end{array}
\right],
\end{equation}
OBS: Ler o Método do jacobiano para descobrir a distribuição!!!!
\end{document}