\documentclass{article}
\usepackage{xcolor}
\usepackage{booktabs}                   % AAB inserido
\usepackage{rotating}                   % AAB inserido

%\usepackage{tikz}
%\usetikzlibrary{shapes,arrows,shadows}
%\usepackage{amsmath,bm,times}
%%%<
%\usepackage{verbatim}
%\usepackage[active,tightpage]{preview}
%\PreviewEnvironment{tikzpicture}
%\setlength\PreviewBorder{5pt}%
%%%>

%\begin{comment}
%Revisão bibliográfica
%\end{comment}


\begin{document}
\section{Artigo \cite{sjx}}
\begin{itemize}
\item Autores J. Shi, H. Jin and Z. Xiao,
\item revista IEEE Access,
\item título \textit{A Novel Hybrid Edge Detection Method for Polarimetric SAR Images,}
\item ano 2020.
\end{itemize}
\subsection{Aplicações}
\begin{itemize}
\item Detecção de bordas.
\item Imagens PolSAR.
\end{itemize}
\subsection{Resumo}
\begin{itemize}
\item 1) A detecção de bordas é realizada utilizando filtros para diminuir speckle.
\item 2) O artigo usa dois detectores estabelecidos na literatura e tenta potencializar o mesmo usando técnica de fusão. 
\item 3) Usa somente um canal.
\item 4) O artigo esta usando limiares para ajustar a fusão.
\item 5) O artigo propõe uma nova maneira de fazer a fusão para os coeficientes SWT, muito interessante. 
\end{itemize}
%############################### template
\section{artigo \cite{wxbzw}}
\begin{itemize}
\item Autores Wei Wang and Deliang Xiang and Yifang Ban and Jun Zhang and Jianwei Wan.
\item revista International Journal of Remote Sensing
\item título \textit{Enhanced edge detection for polarimetric SAR images using a directional span-driven adaptive window}
\item ano 2018.
\end{itemize}
\subsection{Aplicações}
\begin{itemize}
\item Imagens PolSAR.
\item Imagens sintéticas e reais.
\end{itemize}
\subsection{Resumo}
\begin{itemize}
\item Baseado em aplicação de filtros.
\item O artigo propõe um detector de borda aprimorado baseado em uma janela DSDA (\textit{directional span-driven adaptive}).
\end{itemize}

%############################### template
\section{artigo \cite{lzly}}
\begin{itemize}
\item Autores B. {Liu} and Z. {Zhang} and X. {Liu} and W. {Yu}.
\item revista IEEE Geoscience and Remote Sensing Letters
\item título \textit{Edge Extraction for Polarimetric SAR Images Using Degenerate Filter With Weighted Maximum Likelihood Estimation}
\item ano 2014.
\end{itemize}
\subsection{Aplicações}
\begin{itemize}
\item Imagens PolSAR.
\item Imagem simulada e real.
\end{itemize}
\subsection{Resumo}
\begin{itemize}
\item Propõe um filtro baseado \textit{maximum likelihood estimation} chamado de \textit{weighted maximum likelihood estimation (WMLE)}.
\end{itemize}

%############################### template
%\section{artigo \cite{}}
%\begin{itemize}
%\item Autores 
%\item revista 
%\item título \textit{}
%\item ano .
%\end{itemize}
%\subsection{Aplicações}
%\begin{itemize}
%\item
%\item
%\end{itemize}
%\subsection{Resumo}
%\begin{itemize}
%\item
%\item
%\end{itemize}

\section{artigo \cite{tlb}}
\begin{itemize}
\item Autores R. Touzi and A. Lopes and P. Bousquet, 
\item revista IEEE Transactions on Geoscience and Remote Sensing
\item título \textit{A statistical and geometrical edge detector for {SAR} images}
\item ano 1988.
\end{itemize}
\subsection{Aplicações}
\begin{itemize}
\item Detecção de bordas
\item Imagem SAR
\end{itemize}
\subsection{Resumo}
\begin{itemize}
\item 1) Detector de bordas CFAR - Constant False alarm rate
\item 2) O método usa gradiente.
\item 3) Usa limiares
\item 4) Usa a distribuição  Wishart (gaussiana 1-D).
\end{itemize}
\section{artigo \cite{obw}}
\begin{itemize}
\item Autores C. J. Oliver and D. Blacknell and R. G. White
\item revista IEEE Proceedings-Radar, Sonar and Navigation
\item título \textit{Optimum edge detection in SAR}
\item ano 1996.
\end{itemize}
\subsection{Aplicações}
\begin{itemize}
\item Detector de Bordas
\item Imagem SAR
\end{itemize}
\subsection{Resumo}
\begin{itemize}
\item Maximizar a probabilidade total de detectar bordas com uma janela.
\item Maximizar a acurácia com que a posição da borda pode ser determinada.
\item Usar o método da máxima verossimilhança.
\item Usa a pdf Wishart (gaussiana 1-D).
\item Usar limiares. 
\end{itemize}

\section{artigo \cite{flmc}}
\begin{itemize}
\item Autores R. Fjortoft, A. Lopes, P. Marthon and E. Cubero-Castan 
\item revista IEEE Transactions on Geoscience and Remote Sensing
\item título \textit{An optimal multiedge detector for {SAR} image segmentation}
\item ano 1998.
\end{itemize}
\subsection{Aplicações}
\begin{itemize}
\item Detecção de bordas
\item Imagem SAR
\end{itemize}
\subsection{Resumo}
\begin{itemize}
\item É um detector  optimal no sentido d eminimizar o erro médio quadratico MSE. 
\item É um método do tipo CFAR.
\item Tem limiares.
\item Tranbalha com multiplas bordas no ROI.
\item Usa filtros MMSE lineares.
\item Usa métodos de máxima verrossimilhança.
\end{itemize}

%\section{artigo \cite{fyf}}
%\begin{itemize}
%\item Autores X. Fu, H. You and K. Fu
%\item revista IEEE Geoscience and Remote Sensing Letters
%\item título \textit{A Statistical Approach to Detect Edges in {SAR} Images Based on Square Successive Difference of Averages}
%\item ano 2012.
%\end{itemize}
%\subsection{Aplicações}
%\begin{itemize}
%\item Detecção de bordas
%\item Imagem SAR
%\end{itemize}
%\subsection{Resumo}
%\begin{itemize}
%\item É proposto um detector de bordas baseado na diferença sucessiva de quadrados das médias.
%\item CFAR
%\item Limiares no pós processamento.
%\item É um método estatístico e geométrico.
%\item É necessário propor um tamanho de janela.
%\item Usa a distribuição normal gaussiana.
%\item Propõe uma alternativa a direção gradiente.
%\end{itemize}
%############################### template
\section{artigo \cite{law}}
\begin{itemize}
\item Autores J. Lee, T. L. Ainsworth and Y. Wang.
\item revista IEEE International Geoscience and Remote Sensing Symposium (IGARSS-2017)
\item título \textit{A review of polarimetric {SAR} speckle filtering}
\item ano 2017.
\end{itemize}
\subsection{Aplicações}
\begin{itemize}
\item Filtros
\item PolSAR
\end{itemize}
\subsection{Resumo}
\begin{itemize}
\item 1) Avanços recentes de filtros PolSAR.
\item 2) Filtrar o speckle é um procedimento necessário para muitas aplicações PolSAR com intuito de,
	\begin{itemize}	
	\item a)  para reduzir o nível de ruídos.
	\item b)  para formar espalhamento incoerente fazendo média com a matriz de coerência ou a matriz de covariância nos pixels de uma vizinhança.
	\end{itemize}
\item 3) SLC - PolSAR,
\item 4) obs(conclusão) O melhor filtro não existe porque o filtro desenhado é dependente da aplicação ou de preferência pessoal. O filtro pode ser adequado para uma aplicaçã, mas inadequado para outra.
\item 5) citação "Filtros Speckle não é uma ciência exata, e um filtro de speckle perfeito não pode ser alcançado".
\item 6) O objetivo dos filtros de ruído speckle é reduzir o speckle enquanto preserva.
\begin{itemize}
\item a) resolução espacial,
\item b) as propriedades de espalhamento PolSAR.
\item c) as características estatísticas similares ao processo de multiplas visadas das matrizes de coerência ou covariança.
\end{itemize}
\item 7) DLR F-SAR, Pi-SAR2 pode ter resolução de ($0.25 cm\times 0.65 cm$).
\item 8) Para dados com resolução alta o tamanho de cada célula é perto do comprimento de onda do radar. Se isso acontece podemos validar as seguintes afirmações,
\begin{itemize}
\item a) A validade do fenômeno do speckle totalmente desenvolvido.
\item b) Aplicabilidade dos filtros. 
\end{itemize} 
\item 9) A seleção de pixels homogêneos da propriedades de espalhamento é a chave para um efetivo e robusto filtro de speckle.
\end{itemize}


%############################### template
\section{artigo \cite{cgaf}}
\begin{itemize}
\item Autores D. Santana-Cedrés, L. Gomez, L. Alvarez, and A. C. Frery
\item revista IEEE Geoscience and Remote Sensing Letters
\item título \textit{Despeckling PolSAR Images With a Structure Tensor Filter}
\item ano 2020.
\end{itemize}
\subsection{Aplicações}
\begin{itemize}
\item Filtros
\item PolSAR
\item EDP
\end{itemize}
\subsection{Resumo}
\begin{itemize}
\item Cinco classes de filtros importantes
\begin{itemize}
\item a) Filtros locais os quais usa análise de estatística de dados locaispara estimar uma redução do speckle.
\item b) Métodos baseados em EDP's que trabalham na imagem toda.
\item c) Métodos variacionais combinando EDP e estratégias de otimização.
\item d) Aprendizado de máquina e CNN. - \textcolor{red}{olhar o artigo [7] da publicação}
\item e) Filtros não locais
\end{itemize}
\item Usa a distribuíção Wishart.
\end{itemize}
 
\section{artigo \cite{bf}}
\begin{itemize}
\item Autores F. Baselice and G. Ferraioli
\item revista IEEE Geoscience and Remote Sensing Letters
\item título \textit{Statistical Edge Detection in Urban Areas Exploiting {SAR} Complex Data}
\item ano 2012.
\end{itemize}
\subsection{Aplicações}
\begin{itemize}
\item Detecção de bordas
\item Imagens SAR
\end{itemize}
\subsection{Resumo}
\begin{itemize}
\item MRF, Markov Random fields.
\item O método é baseado na exploração conjunta da amplitude e da fase interferométrica.
\item Usa a máxima verossimilhança. 
\end{itemize}

\section{artigo \cite{bac}}
\begin{itemize}
\item Autores J. E. Ball and D. T. Anderson and C. S. Chan
\item revista Journal of Applied Remote Sensing
\item título \textit{Comprehensive survey of deep learning in remote sensing: theories, tools, and challenges for the community}
\item ano 2017.
\end{itemize}
\subsection{Aplicações}
\begin{itemize}
\item Deep learning.
\item Area de sensoriamento remoto
\end{itemize}
\subsection{Resumo}
\begin{itemize}
\item Survey 
\item Tem duas tabelas com apontamentos sobre artigos e aplicações (Boa revisão bibliográfica).
\end{itemize}

%############################### template
\section{artigo \cite{ztmxzxf}}
\begin{itemize}
\item Autores X. X. Zhu and D. Tuia and L. Mou and G. Xia and L. Zhang and F. Xu and F. Fraundorfer
\item revista IEEE Geoscience and Remote Sensing Magazine
\item título \textit{Deep Learning in Remote Sensing: A Comprehensive Review and List of Resources}
\item ano 2017.
\end{itemize}
\subsection{Aplicações}
\begin{itemize}
\item Deep learning em sensoriamento remoto.
\item SAR / PolSAR
\end{itemize}
\subsection{Resumo}
\begin{itemize}
\item Aplicação em classificação da superfície da terra.
\item Alguns artigos comentados sobre machine learning.
 \end{itemize}

%############################### template
\section{artigo \cite{gmbf}}
\begin{itemize}
\item Autores J. Gambini and M. Mejail and J. Jacobo-Berlles and A. C. Frery
\item revista International Journal of Remote Sensing
\item título \textit{Feature Extraction in Speckled Imagery using Dynamic {B}-Spline Deformable Contours under the {G0} Model}
\item ano 2006.
\end{itemize}
\subsection{Aplicações}
\begin{itemize}
\item SAR imagem
\item Introdução do método de deteção de imagens usando a técnica das radiais.
\end{itemize}
\subsection{Resumo}
\begin{itemize}
\item Usa o speckle na análise estatística.
\item B-spline pós-processamento.
\item uso do algoritmo das linhas radiais.
\item Máxima verossimilhança.
\item introdução da imagem simulada de flor.
\end{itemize}

%############################### template
\section{artigo \cite{fbgm}}
\begin{itemize}
\item Autores A. C. Frery, Jacobo-Berlles, J., Gambini, J. and Mejail, M.
\item revista Multidimensional Systems and Signal Processing
\item título \textit{Polarimetric {SAR} Image Segmentation with {B-Splines} and a New Statistical Model}
\item ano 2010.
\end{itemize}
\subsection{Aplicações}
\begin{itemize}
\item PolSAR
\item Modelo estatístico para detecção de bordas.
\end{itemize}
\subsection{Resumo}
\begin{itemize}
\item Artigo smilar ao artigo \cite{gmbf} porém propondo o método de detecão de bordas para imagens PolSAR.
\item Estimativa de parâmetros de uma região usando o método dos momentos. 
\item Uma bom algoritmo para estimativa de erro em imagens simuladas.
\item Interessante resumo teórico no apendice B1.
\end{itemize}

title = "A statistical active contour model for {SAR} image segmentation",
journal = "Image and Vision Computing",
volume = "17",
number = "3",
pages = "213--224",
year = "1999",

%############################### template
\section{artigo \cite{horrit}}
\begin{itemize}
\item Autores Matthew S. Horritt
\item revista Image and Vision Computing
\item título \textit{A statistical active contour model for {SAR} image segmentation}
\item ano 1999.
\end{itemize}
\subsection{Aplicações}
\begin{itemize}
\item SAR imagens
\item Imagens sintéticas e reais.
\end{itemize}
\subsection{Resumo}
\begin{itemize}
\item Modelo de contorno ativo estatístico.
\end{itemize}

%############################### template
\section{artigo \cite{nhfc}}
\begin{itemize}
\item Autores Nascimento, Abraão and Horta, Michelle and Frery, Alejandro and Cintra, Renato.
\item revista Journal of Selected Topics in Applied Earth Observations and Remote Sensing
\item título \textit{Comparing Edge Detection Methods Based on Stochastic Entropies and Distances for {P}ol{SAR} Imagery}
\item ano 2014.
\end{itemize}
\subsection{Aplicações}
\begin{itemize}
\item PolSAR
\item Imagem simulada em duas folhas e imagens reais.
\end{itemize}
\subsection{Resumo}
\begin{itemize}
\item Entropias e distâncias estocásticas
\item Descrição do método das linhas radiais.
\item Aplicação do método da verossimilhança.
\item Distribuíção Wishart.
\item Interessante medida do erro.
\end{itemize}

%############################### template
\section{artigo \cite{bmf_2019}}
\begin{itemize}
\item Autores A. A. {de Borba} and M. {Marengoni} and A. C. {Frery}
\item Congrasso  2019 IEEE Recent Advances in Geoscience and Remote Sensing: Technologies, Standards and Applications (TENGARSS)
\item título \textit{Fusion of Evidences for Edge Detection in {PolSAR} Images}
\item ano 2019.
\end{itemize}
\subsection{Aplicações}
\begin{itemize}
\item PolSAR.
\item Imagem real flevoland.
\end{itemize}
\subsection{Resumo}
\begin{itemize}
\item Distribuíção Wishart para modelar o speckle. Não é usado filtro.
\item Estimativa do parâmetro  $\mu$ usando a média de pontos na amostra, o número de visadas é definida como sendo $4$.
\item Uso de estimativa de máxima verossimilhança para o ponto de transição entre duas regiões com intuito de obter a evidência de borda nos canais de intensidades. 
\item Fusão de evidências de bordas.
\item Métodos de fusão usados, média, SWT, PCA e estatística ROC.
\end{itemize}
%############################### template
\section{artigo \cite{mit}}
\begin{itemize}
\item Autores Mitchell, H.B.
\item editora  Springer Berlin Heidelberg
\item título \textit{Image Fusion: Theories, Techniques and Applications}
\item ano 2010.
\end{itemize}
\subsection{Aplicações}
\begin{itemize}
\item Fusão de imagem.
\end{itemize}
\subsection{Resumo}
\begin{itemize}
\item Fusão de imagens, teoria, técnicas e aplicações.
\end{itemize}


%############################### template
\section{artigo \cite{sg}}
\begin{itemize}
\item Autores A. Salentinig and P. Gamba.
\item revista IEEE Journal of Selected Topics in Applied Earth Observations and Remote Sensing
\item título \textit{IEEE Journal of Selected Topics in Applied Earth Observations and Remote Sensing}
\item ano .
\end{itemize}
\subsection{Aplicações}
\begin{itemize}
\item Imagem SAR
\item Fusão de dados.
\end{itemize}
\subsection{Resumo}
\begin{itemize}
\item Fusão baseada no histograma.
\item Fusão baseada em DWT
\item Filtros de Kalmam de multiescala.
\item Fusão baseada em lógica Fuzzy.
\item Fusão baseadas em operadores lógicos.
\end{itemize}
%############################### template

\section{artigo \cite{xgsh}}
\begin{itemize}
\item Autores Yang Xiang and Sylvain Gubian and Brian Suomela and
      Julia Ho eng
\item revista The R Journal Volume 5/1
\item título \textit{Generalized Simulated Annealing for Efficient Global
      Optimization: the {GenSA} Package for {R}}
\item ano 2013.
\end{itemize}
\subsection{Aplicações}
\begin{itemize}
\item Referência para o método GenSA.
\end{itemize}
%############################### template
\section{artigo \cite{nw}}
\begin{itemize}
\item Autores Jorge Nocedal and Stephen J. Wright.
\item editora Springer
\item título \textit{Numerical Optimization}
\item ano 2006.
\end{itemize}
\subsection{Aplicações}
\begin{itemize}
\item Teoria sobre métodos de otimização, por exemplo o método BFGS usada no artigo.
\end{itemize}
%############################### template
\section{artigo \cite{ht}}
\begin{itemize}
\item Autores Arne Henningsen and Ott Toomet
\item revista Computational Statistics
\item título \textit{maxLik: A package for maximum likelihood estimation in
      {R}}
\item ano 2011.
\end{itemize}
\subsection{Aplicações}
\begin{itemize}
\item Pacote do R que foi usado no artigo com a implementação do método de otimização BFGS.
\end{itemize}

%############################### template
\section{artigo \cite{n_r}}
\begin{itemize}
\item Autores Naidu, V.P.S. and Raol, J.R.
\item revista Defence Science Journal.
\item título \textit{Pixel-level Image Fusion using Wavelets and Principal Component Analysis}
\item ano 2008.
\end{itemize}
\subsection{Aplicações}
\begin{itemize}
\item Técnicas de fusão.
\end{itemize}
\subsection{Resumo}
\begin{itemize}
\item PCA
\item DWT
\item SWT
\end{itemize}   

%############################### template
\section{artigo \cite{gs}}
\begin{itemize}
\item Autores Giannarou, Stamatia and Stathaki, Tania
\item revista EURASIP Journal on Advances in Signal Processing
\item título \textit{Optimal edge detection using multiple operators for image understanding}
\item ano 2011.
\end{itemize}
\subsection{Aplicações}
\begin{itemize}
\item Imagem SAR
\item Estatística ROC
\end{itemize}
\subsection{Resumo}
\begin{itemize}
\item Introdução do método ROC sendo usado para detectores clássicos (Canny, Rothwell, etc..) 
\item Descrição do método dos coeficientes kappa ponderados.
\end{itemize}
%############################### template
\section{artigo \cite{fawcett}}
\begin{itemize}
\item Autores Fawcett, Tom
\item revista Pattern Recognition Letters
\item título \textit{An Introduction to {ROC} Analysis}
\item ano 2006.
\end{itemize}
\subsection{Aplicações}
\begin{itemize}
\item teoria do método ROC.
\end{itemize}
%############################### template
\section{artigo \cite{naidu}}
\begin{itemize}
\item Autores Naidu, V.P.S.
\item revista Defence Science Journal
\item título \textit{Image Fusion Technique using Multi-resolution Singular Value Decomposition}
\item ano 2011.
\end{itemize}
\subsection{Aplicações}
\begin{itemize}
\item Técnicas de fusão.
\end{itemize}
\subsection{Resumo}
\begin{itemize}
\item Descrição do método multiresolution SVD (MSVD).
\end{itemize}
\section{Tabela com as datas dos artigos.}
\begin{table}[hbt]                                                                      \centering
        \caption{Quantidade de artigos por ano.}\label{tab01}
\begin{tabular}{@{}llr@{}} \toprule
        Ano & Quantidades  \\ \midrule                
        2020             & 2 \\          
        2019             & 1\\          
        2018             & 1\\          
        2017             & 3 \\          
        2016             & 1 \\          
        2014             & 2 \\
        2013             & 1 \\
        2012             & 1 \\          
        2011             & 3\\
        2010             & 2 \\
        2008             & 1\\          
        2006             & 3 \\
        1999             & 1\\          
        1998             & 1\\         
        1996             & 1 \\
        1988             & 1  \\     \bottomrule
\end{tabular}
\end{table}
\section{Tabela com o número de artigos por revista.}
%\begin{table}[hbt]
\begin{sidewaystable}                                                                      \centering
        \caption{Quantidade de artigos por revista.}\label{tab02}
\begin{tabular}{@{}llr@{}} \toprule
        Ano & Quantidades  \\ \midrule                
     IEEE Geoscience and Remote Sensing Letters                & 4 \\
     International Journal of Remote Sensing                   & 2  \\
     IEEE Transactions on Geoscience and Remote Sensing        & 2 \\
     Journal of Selected Topics in Applied Earth Observations and Remote Sensing      & 2 \\
     Springer (livros)                                         & 2\\
     Defence Science Journal                                   & 2\\
     IEEE Access                                               & 1 \\
     IEEE Proceedings-Radar, Sonar and Navigation              & 1 \\
     IEEE International Geoscience and Remote Sensing Symposium (IGARSS-2017) & 1\\
     Journal of Applied Remote Sensing                         & 1\\
     Multidimensional Systems and Signal Processing            & 1\\
     Image and Vision Computing                                & 1 \\
     2019 IEEE Recent Advances in Geoscience and Remote Sensing: Technologies, Standards and Applications (TENGARSS) & 1\\
     The R Journal                                             & 1\\
     Computational Statistical                                 & 1 \\
      EURASIP Journal on Advances in Signal Processing         & 1 \\
      Pattern Recognition Letters                              & 1  \\     \bottomrule
\end{tabular}
\end{sidewaystable}  


%############################### template
%\section{artigo \cite{}}
%\begin{itemize}
%\item Autores 
%\item revista 
%\item título \textit{}
%\item ano .
%\end{itemize}
%\subsection{Aplicações}
%\begin{itemize}
%\item
%\item
%\end{itemize}
%\subsection{Resumo}
%\begin{itemize}
%\item
%\item
%\end{itemize}



% Define the layers to draw the diagram
%\pgfdeclarelayer{background}
%\pgfdeclarelayer{foreground}
%\pgfsetlayers{background,main,foreground}

% Define block styles used later

%\tikzstyle{sensor}=[draw, fill=blue!20, text width=5em, 
%    text centered, minimum height=2.5em,drop shadow]
%\tikzstyle{ann} = [above, text width=5em, text centered]
%\tikzstyle{wa} = [sensor, text width=10em, fill=red!20, 
%    minimum height=6em, rounded corners, drop shadow]
%\tikzstyle{sc} = [sensor, text width=13em, fill=red!20, 
%    minimum height=10em, rounded corners, drop shadow]

% Define distances for bordering
%\def\blockdist{2.3}
%\def\edgedist{2.5}

%\begin{tikzpicture}
%	\node (wa) [wa]  {$V=\sum_{i=1}^{N}E_i$};
%	\path (wa.west)+(-3.2,1.5) node (e1) [sensor] {$E_1$};
%    \path (wa.west)+(-3.2,0.5) node (e2)[sensor] {$E_2$};
%    \path (wa.west)+(-3.2,-1.0) node (dots)[ann] {$\vdots$}; 
%    \path (wa.west)+(-3.2,-2.0) node (e3)[sensor] {$E_N$};    
%   
%    \path (wa.east)+(3.2,1.5) node (m1) [sensor] {$M_1$};
%    \path (wa.east)+(3.2,0.5) node (m2) [sensor] {$M_2$};
%    \path (wa.east)+(3.2,-1.0) node (dots)[ann] {$\vdots$}; 
%    \path (wa.east)+(3.2,-2.0) node (m3) [sensor] {$M_N$};
%
%    \path [draw, ->] (e1.east) -- node [above] {} 
%        (wa.160) ;
%    \path [draw, ->] (e2.east) -- node [above] {} 
%        (wa.180);
%    \path [draw, ->] (e3.east) -- node [above] {} 
%        (wa.200);
%	\path [draw, ->] (wa.east) -- node [above] {\tiny{$CT_1$}} 
%        (m1.west);
%	\path [draw, ->] (wa.east) -- node [above] {\tiny{$CT_2$}} 
%        (m2.west);
%	\path [draw, ->] (wa.east) -- node [right] {\tiny{$CT_N$}} 
%        (m3.west);
%               
%    \path (wa.south) +(0,-\blockdist) node (asrs) {Estrutura geral da fusão de evidência proposta};
%  
%    \begin{pgfonlayer}{background}
%        \path (e1.west |- e1.north)+(-0.5,0.3) node (a) {};
%        \path (wa.south -| wa.east)+(+0.5,-0.3) node (b) {};
%        \path (m3.east |- m3.east)+(+0.5,-0.5) node (c) {};
%       %   
%        \path[fill=yellow!20,rounded corners, draw=black!50, dashed]
%            (a) rectangle (c);           
%       %     
%    \end{pgfonlayer}
 %   
%
%\end{tikzpicture}

\bibliographystyle{IEEEtran}
\bibliography{../Text/bibliografia}
\end{document}
