\section{Detecção de bordas em imagens PolSAR}\label{cap_det_borda}

Na literatura encontramos uma grande oferta de métodos clássicos para detectar bordas, por exemplo Sobel, Canny, Laplaciano da gaussiana(LoG) e LoG piramidal. Os métodos clássicos de detecção de bordas são construídos assumindo que o ruído é aditivo, tornando esses métodos ineficientes para aplicação em imagens PolSAR, entretanto, nas imagens PolSAR o ruído é multiplicativo. 

Na corrente seção conceitos baseados nos artigos \citet{nhfc, gmbf} serão introduzidos e proporemos modificações para os métodos de detecção de borda em imagens PolSAR com multiplas visadas. A ideia principal é detectar o ponto de transição em uma faixa tão fina quanto possível entre duas regiões da imagem. O ponto de transição é definido como uma evidência de borda. Os ruídos nesse tipo de imagens são do tipo \textit{speckle}, os mesmos têm natureza multiplicativa, tornando a detecção de bordas em imagens SAR uma tarefa desafiadora.

Podemos indicar que o problema de detecção de bordas, pode ser resumido em três importantes aspectos:
\begin{enumerate}
	\item o procedimento para detecção;
	\item a determinação de uma posição mais acurada da posição da borda;
	\item a especificação de tamanho para uma janela (pode ser uma janela quadrada ou em uma faixa de dados). 
\end{enumerate}

O tamanho da janela pode influenciar em alguns aspectos como por exemplo, em uma janela pequena pode não conter informações para identificar a presença de bordas, ou em janelas maiores podem obter informações para mais de uma borda. Assim o tamanho de janela ideal é aquele que contém as informações para detecção de uma borda. Vamos assumir que há uma borda na janela fornecida pela seleção inicial, quando forem realizados os testes computacionais.

Em linhas gerais seguiremos as seguintes afirmações: 
\begin{enumerate}
	\item tentar encontrar finas faixas de dados, idealmente do tamanho de um pixel;
	\item será usado o método de máxima verossimilhança;
	\item de que maneira a detecção trabalha em diferentes canais da imagem PolSAR. 
\end{enumerate}

De uma maneira geral, a ideia se baseia em encontrar um ponto de transição em uma faixa de dados, o qual é considerado uma estimativa de posicionamento da borda, isto é, uma evidência de borda. A evidência de borda é encontrada em um processo de otimização. 

As metodologias de detecção de bordas ocorrem em diversos estágios, abaixo enumeramos os estágios:
\begin{enumerate}
	\item identificar o centroide de uma região de interesse (ROI) de maneira automática, semiautomática ou manual;
	\item construir raios do centroide para fora da área de interesse;
	\item coletar dados em uma vizinhança em torno dos raios;
	\item detectar pontos na faixa de dados os quais fornecem evidências de mudanças de propriedades estatística, ou seja, uma transição que define uma evidência de borda;
	\item definir o contorno usando um método de interpolação entre os pontos de transição, por exemplo, as B-Splines, ou o método dos quadrados mínimos \textbf{MMQ}.
\end{enumerate}

Inicialmente, escolhemos uma região de interesse (ROI) $\mathbf{R}$ com centroide $\mathbf{C}$ e traçamos raios iniciando em $\mathbf{C}$ e indo até um ponto de controle $\mathbf{P}_i$, com $i=1,2,\dots, \mathbf{S}$, estes pontos de controle estão fora da região $\mathbf{R}$. Teremos $\mathbf{S}$ raios resultantes representados por ${\bf s^{(i)}}=\overline{\mathbf{CP}_i}$ com ângulos $\epsilon_{i}=\angle ({\bf s_{(i)}},{\bf s_{(i+1)}})$. 

Os raios serão convertidos sobre pixeis usando o algoritmo {\it Bresenham's midpoint line algorithm}, esse algoritmo fornece uma fina representação digital para os raios. Portanto, em cada raio vamos assumir que os dados seguem uma distribuição complexa Wishart com sua respectiva função de distribuição dada por (\ref{cap_acf_9}) e denotada por $W(\Sigma,L)$.

A faixa de dados coletada no $\mathit{i}$ - ésimo raio $\mathbf{s^{i}}$, com $i$ variando em $i=1,2,\dots, S$, contém $N^{(i)}$ pixeis. Para cada pixel $\mathit{k}$ em uma dada faixa $\mathit{i}$, a mesma pode ser descrita pelo resultado da matriz $Z_{k}^{(i)}$ sendo esta uma distribuição de Wishart, portanto podemos representar cada pixel como uma distribuição,  
\begin{equation}\label{cap_acf_13}
 \left\{
\begin{array}{cl}
	Z_{k}^{(i)}\sim W(\Sigma_{A}^{(i)},L_{A}^{(i)}),& \mbox{para}\quad k=1,\dots,j^{(i)} , \\
	Z_{k}^{(i)}\sim W(\Sigma_{B}^{(i)},L_{B}^{(i)}),& \mbox{para}\quad k=j^{(i)} + 1,\dots,N^{(i)}.  \\
\end{array}
\right.
\end{equation}

Podemos definir cada faixa composta de dois tipos de amostras, e, cada tipo obedece uma lei de Wishart complexa com diferentes parâmetros. Vamos assumir que o número de visadas é constante para todas as faixas. A ideia principal é encontrar uma evidência de borda $j^{(i)}$ em cada faixa, ao longo do raio ${\bf s^{(i)}}$, a evidência de borda ou ponto de transição representa um pixel de transição entre as duas amostras. O modelo proposto em (\ref{cap_acf_13}) assume que existir uma transição ocorrendo ao longo da faixa $\bf s^{(i)}$. Sem perda de generalidade na continuidade do trabalho será omitido o índice $(i)$ focando nossa análise em uma única faixa.
% *******************************************************
\section{Estimativa por Máxima verossimilhança (\textbf{MLE})}\label{cap_acf_sec3}

O conceito de verossimilhança é importante em estatística, pois descreve de maneira precisa as informações sobre os parâmetros do modelo estatístico que representa os dados observados. De maneira geral, a estimativa por máxima verossimilhança (\textbf{MLE}) é um método que, tendo um conjunto de dados e um modelo estatístico, estima os valores dos parâmetros do modelo estatísticos com intuito de maximizar uma função de probabilidade dos dados. 

\section{Função de verossimilhança}

Suponha $\mathbf{X}=(X_1,X_2,\dots,X_n)^T$ um vetor randômico distribuído de acordo com a $\mathbf{p.d.f}$ função densidade de probabilidade $f(\mathbf{x},\mathbf{\theta})$ com parâmetros $\mathbf{\theta}=(\theta_1,\dots,\theta_d)^T$ no espaço dos parâmetros $\Theta$. Definimos  a \textbf{função de verossimilhança} cuja amostra é 
\begin{equation}\label{cap_acf_14}
    L(\theta;\mathbf{X}) = \prod_{i=1}^{n}f(x_i;\theta), \\
\end{equation}
e a função logarítmica de verossimilhança a qual podemos chamar de  \textbf{função de log-verossimilhança} é a soma
\begin{equation}\label{cap_acf_15}
	l(\theta;\mathbf{X})= \ln(L(\theta;\mathbf{X})) = \sum_{i=1}^{n}\ln(f(x_i;\theta)). \\
\end{equation}

Podemos definir o método da \textbf{estimativa de máxima verossimilhança} (\textbf{MLE}) de $\theta$ como sendo o vetor $\widehat{\theta}$ tal que $L(\widehat{\theta};\mathbf{x})\ge L(\theta;\mathbf{x})$ para todo $\theta$ no espaço dos parâmetros $\Theta$. De maneira simplificada a \textbf{estimativa de máxima verossimilhança} pode ser escrita por 
\begin{equation}\label{cap_acf_16}
    \widehat{\theta}= \text{arg}\,\max\limits_{\theta\in\Theta}L(\theta;\mathbf{x}),\\
\end{equation}
ou de maneira similar 
\begin{equation}\label{cap_acf_17}
    \widehat{\theta}= \text{arg}\,\max\limits_{\theta\in\Theta}l(\theta;\mathbf{x}).\\
\end{equation}

\section{Estimativa de máxima verossimilhança aplicada a distribuíção Wishart}

Nesta seção vamos usar o método de máxima verossimilhança aplicado na distribuição Wishart. Suponha $\mathbf{Z}=(\mathbf{Z}_1,\mathbf{Z}_2,\dots,\mathbf{Z}_N)^T$ um vetor randômico distribuído de acordo com a $\mathbf{p.d.f}$ função densidade de probabilidade (\ref{cap_acf_9}) com parâmetros $\Sigma=\{\mathbf{\Sigma_A}, \mathbf{\Sigma_B\}}$ e $L$.

A \textbf{função de verossimilhança} da amostra descrita por (\ref{cap_acf_13}) é dada pela equação produtório das funções de densidade respectivamente associadas a cada amostra
\begin{equation}\label{cap_acf_18}
	L(j)=\prod_{k=1}^{j}f_{\mathbf{Z}}(\mathbf{Z}_{k}^{'};\mathbf{\Sigma_{A}},L) \prod_{k=j+1}^{N}f_{\mathbf{Z}}(\mathbf{Z}_{k}^{'};\mathbf{\Sigma_{B}},L), \\
\end{equation}
onde $\mathbf{Z}_{k}^{'}$ é uma possível aproximação da matriz randômica descrita em (\ref{cap_acf_13}).

Usando a equação (\ref{cap_acf_12}) e propriedades de logaritmos natural teremos para cada termo do produtório (\ref{cap_acf_18}) uma \textbf{função de log-verossimilhança}. Assim, encontramos uma função dependente de $j$
\begin{equation}\label{cap_acf_19}
	l(j)=\ln L(j)=\sum_{k=1}^{j}\ln f_{\mathbf{Z}}(\mathbf{Z}_{k}^{'};\mathbf{\Sigma_{A}},L)+ \sum_{k=j+1}^{N}\ln f_{\mathbf{Z}}(\mathbf{Z}_{k}^{'};\mathbf{\Sigma_{B}},L).
\end{equation}

Nesse momento, podemos realizar  manipulações algébricas na função distribuição em cada termo do somatório conforme (\ref{cap_acf_12}) e substituir nas duas parcelas da equação (\ref{cap_acf_19}) levando em consideração que as duas amostras são diferentes, desta forma
\begin{equation*}
\begin{array}{rcl}
	l(j)&=&\sum_{k=1}^{j}\left[mL\ln{\left(L\right)}+(L-m)\ln{\left(|\mathbf{Z}_{k}^{'}|\right)}- L\ln{\left(|\mathbf{\Sigma_{A}}|\right)}-\ln{\left(\Gamma_m(L)\right)}-L tr(\mathbf{\Sigma_{A}}^{-1}\mathbf{Z}_{k}^{'}))\right], \\
	&+&\sum_{k=j+1}^{N}\left[mL\ln{\left(L\right)}+(L-m)\ln{\left(|\mathbf{Z}_{k}^{'}|\right)}- L\ln{\left(|\mathbf{\Sigma_{B}}|\right)}-\ln{\left(\Gamma_m(L)\right)}-L tr(\mathbf{\Sigma_{B}}^{-1}\mathbf{Z}_{k}^{'})\right], \\
	l(j)&=&\sum_{k=1}^{N}\left[mL\ln{\left(L\right)}-\ln{\left(\Gamma_m(L)\right)}\right]-\sum_{k=1}^{j}\left[ L\ln{\left(|\mathbf{\Sigma_{A}}|\right)}-L tr(\mathbf{\Sigma_{A}}^{-1}\mathbf{Z}_{k}^{'}))\right], \\
	&+&\sum_{k=1}^{N}\left[(L-m)\ln{\left(|\mathbf{Z}_{k}^{'}|\right)}\right]- \sum_{k=j+1}^{N}\left[L\ln{\left(|\mathbf{\Sigma_{B}}|\right)}-L tr(\mathbf{\Sigma_{B}}^{-1}\mathbf{Z}_{k}^{'})\right], \\
	l(j)&=&N\left[mL\ln{\left(L\right)}-\ln{\left(\Gamma_m(L)\right)}\right]-L\left[j\ln{\left(|\mathbf{\Sigma_{A}}|\right)} +\sum_{k=1}^{j}tr(\mathbf{\Sigma_{A}}^{-1}\mathbf{Z}_{k}^{'})\right], \\
	&+&(L-m)\sum_{k=1}^{N}\ln{\left(|\mathbf{Z}_{k}^{'}|\right)}-L\left[(N-j)\ln{\left(|\mathbf{\Sigma_{B}}|\right)}+ \sum_{k=j+1}^{N}tr(\mathbf{\Sigma_{B}}^{-1}\mathbf{Z}_{k}^{'})\right], \\
	l(j)&=&N\left[mL\ln{\left(L\right)}-\ln{\left(\Gamma_m(L)\right)}\right]-L\left[j\ln{\left(|\mathbf{\Sigma_{A}}|\right)} +(N-j)\ln{\left(|\mathbf{\Sigma_{B}}|\right)}\right], \\
	&+&(L-m)\sum_{k=1}^{N}\ln{\left(|\mathbf{Z}_{k}^{'}|\right)}-L\left[\sum_{k=1}^{j}tr(\mathbf{\Sigma_{A}}^{-1}\mathbf{Z}_{k}^{'})+ \sum_{k=j+1}^{N}tr(\mathbf{\Sigma_{B}}^{-1}\mathbf{Z}_{k}^{'})\right]. \\
\end{array}
\end{equation*}

A matriz $\Sigma$ pode ser encontrada usando o estimador de máxima verossimilhança denotado por $\widehat{\Sigma}$ de acordo com a referência \citep{good}. A equação (\ref{cap_acf_20}) representa duas estimativa para a matriz de covariância $\Sigma$ que dependem da posição $j$
\begin{equation}\label{cap_acf_20}
\widehat{\Sigma_{I}}(j) = \left\{
\begin{array}{lc}
	j^{-1}\sum_{k=1}^{j}\mathbf{Z}_{k}  & \mbox{se}\quad I=A,  \\
        (N-j)^{-1}\sum_{k=j+1}^{N}\mathbf{Z}_{k} & \mbox{se}\quad I=B. \\
\end{array}
\right.
\end{equation}

Usando a equação (\ref{cap_acf_20}) podemos substituir na equação acima e continuar a manipulação algébrica
\begin{equation*}
\begin{array}{rcl}
	l(j)&=&N\left[mL\ln{\left(L\right)}-\ln{\left(\Gamma_m(L)\right)}\right]-L\left[j\ln{\left(|\mathbf{\Sigma_{A}}|\right)} +(N-j)\ln{\left(|\mathbf{\Sigma_{B}}|\right)}\right], \\
	&+&(L-m)\sum_{k=1}^{N}\ln{\left(|\mathbf{Z}_{k}^{'}|\right)}-L\left[\sum_{k=1}^{j}tr(\mathbf{\Sigma_{A}}^{-1}\mathbf{Z}_{k}^{'})+ \sum_{k=j+1}^{N}tr(\mathbf{\Sigma_{B}}^{-1}\mathbf{Z}_{k}^{'})\right], \\
	l(j)&=&N\left[mL\ln{\left(L\right)}-\ln{\left(\Gamma_m(L)\right)}\right]-L\left[j\ln{\left(|\mathbf{\Sigma_{A}}|\right)} +(N-j)\ln{\left(|\mathbf{\Sigma_{B}}|\right)}\right], \\
	&+&(L-m)\sum_{k=1}^{N}\ln{\left(|\mathbf{Z}_{k}^{'}|\right)}-L\left[tr\left(\sum_{k=1}^{j}\mathbf{\Sigma_{A}}^{-1}\mathbf{Z}_{k}^{'}\right)+tr\left( \sum_{k=j+1}^{N}\mathbf{\Sigma_{B}}^{-1}\mathbf{Z}_{k}^{'}\right)\right], \\
	l(j)&=&N\left[mL\ln{\left(L\right)}-\ln{\left(\Gamma_m(L)\right)}\right]-L\left[j\ln{\left(|\mathbf{\Sigma_{A}}|\right)} +(N-j)\ln{\left(|\mathbf{\Sigma_{B}}|\right)}\right], \\
	&+&(L-m)\sum_{k=1}^{N}\ln{\left(|\mathbf{Z}_{k}^{'}|\right)}-L\left[tr\left(\mathbf{\Sigma_{A}}^{-1}\sum_{k=1}^{j}\mathbf{Z}_{k}^{'}\right)+tr\left( \mathbf{\Sigma_{B}}^{-1}\sum_{k=j+1}^{N}\mathbf{Z}_{k}^{'}\right)\right], \\
	l(j)&=&N\left[mL\ln{\left(L\right)}-\ln{\left(\Gamma_m(L)\right)}\right]-L\left[j\ln{\left(|\mathbf{\Sigma_{A}}|\right)} +(N-j)\ln{\left(|\mathbf{\Sigma_{B}}|\right)}\right], \\
	&+&(L-m)\sum_{k=1}^{N}\ln{\left(|\mathbf{Z}_{k}^{'}|\right)}-L\left[mj+(N-j)m\right], \\
	l(j)&=&N\left[mL\ln{\left(L\right)}-\ln{\left(\Gamma_m(L)\right)}\right]-L\left[j\ln{\left(|\mathbf{\Sigma_{A}}|\right)} +(N-j)\ln{\left(|\mathbf{\Sigma_{B}}|\right)}\right], \\
	&+&(L-m)\sum_{k=1}^{N}\ln{\left(|\mathbf{Z}_{k}^{'}|\right)}-LNm, \\
	l(j)&=&N\left[-mL(1-\ln{\left(L\right)})-\ln{\left(\Gamma_m(L)\right)}\right]-L\left[j\ln{\left(|\mathbf{\Sigma_{A}}|\right)} +(N-j)\ln{\left(|\mathbf{\Sigma_{B}}|\right)}\right], \\
	&+&(L-m)\sum_{k=1}^{N}\ln{\left(|\mathbf{Z}_{k}^{'}|\right)}, \\
\end{array}
\end{equation*}
portanto, 
\begin{equation}\label{cap_acf_21}
\begin{array}{rcl}
	l(j)&=&N\left[-mL(1-\ln{\left(L\right)})-\ln{\left(\Gamma_m(L)\right)}\right]-L\left[j\ln{\left(|\mathbf{\widehat{\Sigma}}_{A}(j)|\right)} +(N-j)\ln{\left(|\mathbf{\widehat{\Sigma}}_{B}(j)|\right)}\right], \\
	&+&(L-m)\sum_{k=1}^{N}\ln{\left(|\mathbf{Z}_{k}^{'}|\right)}. \\
\end{array}
\end{equation}

O estimador de máxima verossimilhança $\widehat{\jmath}_{ML}$ é uma evidência de borda por representar uma aproximação da transição de região e pode ser calculado pelo método de maximização
\begin{equation}\label{cap_acf_22}
\begin{array}{rcl}
	\widehat{\jmath}_{ML}&=&\text{arg}\max\limits_{j}l(j).  \\
\end{array}
\end{equation}

\subsection{Estimativa de máxima verossimilhança aplicada na densidade de magnitude do produto $\mathbf{S}_i$ e $\mathbf{S}_j$}
A \textbf{função de verossimilhança} da amostra descrita por (\ref{}) é dada pela produtório das funções de densidade respectivamente associadas a cada amostra gerando a a função $L$
\begin{equation}
	L(j)=\prod_{k=1}^{j}p(\mathbf{Z}_{k}^{'}) \prod_{k=j+1}^{N}p(\mathbf{Z}_{k}^{'}), \\
\end{equation}
onde $\mathbf{Z}_{k}^{'}$ é uma possível aproximação da matriz randômica descrita em (ref).
Usando a equação (\ref) e propriedades de logaritmos natural teremos para cada termo do produtório (\ref{cap_acf_18}) uma \textbf{função de log-verossimilhança}. Assim, encontramos uma função dependente de $j$
\begin{equation}
\begin{array}{rcl}
	l(j)=\ln L(j)&=&\sum_{k=1}^{j}\ln p(\mathbf{Z}_{k}^{'})\\
	             &+&\sum_{k=j+1}^{N}\ln p(\mathbf{Z}_{k}^{'}).
\end{array}
\end{equation}

Substituindo a função de densidade e realizando as operações algébricas teremos
\begin{equation}
\begin{array}{rcl}
	l(j)=\ln L(j)&=&N\ln 4+N(L+1)\ln L\\
	             &-&N\ln\Gamma(L)-N\ln(1-|\rho|^2)\\
	             &+&L\sum_{k=1}^{j}\ln\xi +L\sum_{k=j+1}^{N}\ln\xi \\
	             &+&\sum_{k=1}^{j}\ln I_0\left(\frac{2|\rho|L\xi}{1-|\rho|^2}\right) +\sum_{k=j+1}^{N} \ln I_0\left(\frac{2|\rho|L\xi}{1-|\rho|^2}\right)        \\
	             &+&\sum_{k=1}^{j}\ln K_{L-1}\left(\frac{2L\xi}{1-|\rho|^2}\right) +\sum_{k=j+1}^{N} \ln K_{L-1}\left(\frac{2L\xi}{1-|\rho|^2}\right)         \\
\end{array}
\end{equation}


\subsection{Estimativa de máxima verossimilhança aplicada na densidade de magnitude do produto $\mathbf{S}_i$ e $\mathbf{S}_j$}
As funções de verossimilhança das amostras descritas por (\ref{eq_10}) e (\ref{eq_11}) serão usadas juntamente com a função de densidade (pdf) para a magnitude do produto $\mathbf{S}_i$ e $\mathbf{S}_j$. Partindo de 
\begin{equation*}
\begin{array}{rcl}
	l(j)=\ln L(j)&=&\sum_{k=1}^{j}\ln f_{\mathbf{Z}}(\mathbf{Z}_{k}^{'};\mathbf{\Sigma_{A}},L)\\
	             &+&\sum_{k=j+1}^{N}\ln f_{\mathbf{Z}}(\mathbf{Z}_{k}^{'};\mathbf{\Sigma_{B}},L).
\end{array}
\end{equation*}
e usando a distribuição (\ref{eq_08}) para duas amostras diferentes $A$ e $B$ teremos,
\begin{equation}
\begin{array}{rcl}
	l(j)=\ln L(j)&=&N\ln 4+N(L+1)\ln L\\
	             &-&N\ln\Gamma(L)-N\ln(1-|\rho|^2)\\
	             &+&L\sum_{k=1}^{j}\ln g_k+L\sum_{k=j+1}^{N}\ln g_k \\
	             &-&(L+1)\sum_{k=1}^{j}\ln h_k\\
	             &-&(L+1)\sum_{k=j+1}^{N}\ln h_k \\
	             &+&\sum_{k=1}^{j}\ln I_0\left(\frac{2|\rho|L\xi}{1-|\rho|^2}\right)\\ 
	             &+&\sum_{k=j+1}^{N} \ln I_0\left(\frac{2|\rho|L\xi}{1-|\rho|^2}\right)\\
	             &+&\sum_{k=1}^{j}\ln K_{L-1}\left(\frac{2L\xi}{1-|\rho|^2}\right)\\
	             &+&\sum_{k=j+1}^{N}\ln K_{L-1}\left(\frac{2L\xi}{1-|\rho|^2}\right)\\
	             
\end{array}
\end{equation}
