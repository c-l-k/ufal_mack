\documentclass[10pt,a4paper]{article}
\usepackage[english,brazil]{babel}
\usepackage[utf8]{inputenc}

\begin{document}
\nocite{lee94}

\section{Estudo da bibliografia}

Este arquivo serve para fazer apontamentos acerca da bibliografia indicada/pesquisada.

\subsection{Estudo do artigo  \cite{lee94}}.

A matriz de espalhamento complexa {\bf S} é definida por

$$
{\bf S} = \left[
\begin{array}{cc}
	S_{hh}   & S_{hv}   \\
	S_{vh}   & S_{vv}   \\
\end{array}
\right].
$$

Por facilidade usaremos o fato de ser um {\it reciprocal medium}, isto é, $S_{hv}=S_{vh}$
$$
{\bf S} = \left[
\begin{array}{c}
	S_{vv}      \\
	S_{vh}     \\
	S_{hh}      \\
\end{array}
\right].
$$

De acordo com \cite{goodman1963} a distribuíção gaussiana complexa multivariada pode modelar adequadamente o comportamento estatístico de $\bf S$. Isto é chamado de {\it single-look PolSar data representation} e podemos definir o vetor de espalhamento por $S=[S_1,S_2,\dots,S_p]^t$. 



\subsection{Estudo do artigo  \cite{goodman1963}}








\bibliographystyle{unsrt}
\bibliography{/home/aborba/git_ufal_mack/Text/bibliografia} 
\end{document}
