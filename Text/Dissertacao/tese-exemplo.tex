% Arquivo LaTeX de exemplo de dissertação/tese a ser apresentados à CPG do IME-USP
% 
% Versão 5: Sex Mar  9 18:05:40 BRT 2012
%
% Criação: Jesús P. Mena-Chalco
% Revisão: Fabio Kon e Paulo Feofiloff
%  
% Obs: Leia previamente o texto do arquivo README.txt
\documentclass[11pt,twoside,a4paper]{book}
% ---------------------------------------------------------------------------- %
% Pacotes 
%\usepackage[T1]{fontenc}
\usepackage[utf8]{inputenc}
\usepackage[brazil]{babel}
%\usepackage[latin1]{inputenc}
\usepackage[pdftex]{graphicx}           % usamos arquivos pdf/png como figuras
\usepackage{setspace}                   % espaçamento flexível
\usepackage{indentfirst}                % indentação do primeiro parágrafo
\usepackage{makeidx}                    % índice remissivo
\usepackage[nottoc]{tocbibind}          % acrescentamos a bibliografia/indice/conteudo no Table of Contents
\usepackage{courier}                    % usa o Adobe Courier no lugar de Computer Modern Typewriter
\usepackage{type1cm}                    % fontes realmente escaláveis
\usepackage{listings}                   % para formatar código-fonte (ex. em Java)
\usepackage[font=small,format=plain,labelfont=bf,up,textfont=it,up]{caption}
\usepackage[usenames,svgnames,dvipsnames]{xcolor}
\usepackage[a4paper,top=2.54cm,bottom=2.0cm,left=2.0cm,right=2.54cm]{geometry} % margens
\usepackage{titletoc}
\usepackage{amsmath,amssymb}            % AAB inserido
\usepackage{booktabs}                   % AAB inserido
\usepackage{siunitx}                    % AAB inserido
\usepackage{rotating}                   % AAB inserido
\usepackage{tikz}                       % AAB inserido
\usetikzlibrary{shapes,arrows,shadows}  % AAB inserido
                                        % AAB inserido
\usepackage[caption=false,font=normalsize,labelfont=sf,textfont=sf]{subfig}
%\usepackage[caption=false,font=footnotesize]{subfig}
%\usepackage{amsmath,bm,times}           % AAB inserido
%%%<
%\usepackage[alf,abnt-etal-cite=2]{abntcite} % AAB verificar as citacoes
%\usepackage[bf,small,compact]{titlesec} % cabeçalhos dos títulos: menores e compactos
%\usepackage[fixlanguage]{babelbib}
%\usepackage[pdftex,plainpages=false,pdfpagelabels,pagebackref,colorlinks=true,citecolor=black,linkcolor=black,urlcolor=black,filecolor=black,bookmarksopen=true]{hyperref} % links em preto
\usepackage[pdftex,plainpages=false,pdfpagelabels,pagebackref,colorlinks=true,citecolor=DarkGreen,linkcolor=NavyBlue,urlcolor=DarkRed,filecolor=green,bookmarksopen=true]{hyperref} % links coloridos
\usepackage[all]{hypcap}                    % soluciona o problema com o hyperref e capitulos
\usepackage[round,sort,nonamebreak]{natbib} % citação bibliográfica textual(plainnat-ime.bst)
\fontsize{60}{62}\usefont{OT1}{cmr}{m}{n}{\selectfont}

% ---------------------------------------------------------------------------- %
% Cabeçalhos similares ao TAOCP de Donald E. Knuth
\usepackage{fancyhdr}
\pagestyle{fancy}
\fancyhf{}
\renewcommand{\chaptermark}[1]{\markboth{\MakeUppercase{#1}}{}}
\renewcommand{\sectionmark}[1]{\markright{\MakeUppercase{#1}}{}}
\renewcommand{\headrulewidth}{0pt}
\newtheorem{definition}{Definição}[section]
\DeclareMathOperator{\traco}{tr}

% ---------------------------------------------------------------------------- %
\graphicspath{{./figuras/}}             % caminho das figuras (recomendável)
\frenchspacing                          % arruma o espaço: id est (i.e.) e exempli gratia (e.g.) 
\urlstyle{same}                         % URL com o mesmo estilo do texto e não mono-spaced
\makeindex                              % para o índice remissivo
\raggedbottom                           % para não permitir espaços extra no texto
\fontsize{60}{62}\usefont{OT1}{cmr}{m}{n}{\selectfont}
\cleardoublepage
\normalsize

% ---------------------------------------------------------------------------- %
% Opções de listing usados para o código fonte
% Ref: http://en.wikibooks.org/wiki/LaTeX/Packages/Listings
\lstset{ %
language=Java,                  % choose the language of the code
basicstyle=\footnotesize,       % the size of the fonts that are used for the code
numbers=left,                   % where to put the line-numbers
numberstyle=\footnotesize,      % the size of the fonts that are used for the line-numbers
stepnumber=1,                   % the step between two line-numbers. If it's 1 each line will be numbered
numbersep=5pt,                  % how far the line-numbers are from the code
showspaces=false,               % show spaces adding particular underscores
showstringspaces=false,         % underline spaces within strings
showtabs=false,                 % show tabs within strings adding particular underscores
frame=single,	                % adds a frame around the code
framerule=0.6pt,
tabsize=2,	                    % sets default tabsize to 2 spaces
captionpos=b,                   % sets the caption-position to bottom
breaklines=true,                % sets automatic line breaking
breakatwhitespace=false,        % sets if automatic breaks should only happen at whitespace
escapeinside={\%*}{*)},         % if you want to add a comment within your code
backgroundcolor=\color[rgb]{1.0,1.0,1.0}, % choose the background color.
rulecolor=\color[rgb]{0.8,0.8,0.8},
extendedchars=true,
xleftmargin=10pt,
xrightmargin=10pt,
framexleftmargin=10pt,
framexrightmargin=10pt
}

% ---------------------------------------------------------------------------- %
% Corpo do texto
\begin{document}
\frontmatter 
% cabeçalho para as páginas das seções anteriores ao capítulo 1 (frontmatter)
\fancyhead[RO]{{\footnotesize\rightmark}\hspace{2em}\thepage}
\setcounter{tocdepth}{2}
\fancyhead[LE]{\thepage\hspace{2em}\footnotesize{\leftmark}}
\fancyhead[RE,LO]{}
\fancyhead[RO]{{\footnotesize\rightmark}\hspace{2em}\thepage}

\onehalfspacing  % espaçamento

% ---------------------------------------------------------------------------- %
% CAPA
% Nota: O título para as dissertações/teses do IME-USP devem caber em um 
% orifício de 10,7cm de largura x 6,0cm de altura que há na capa fornecida pela SPG.
\thispagestyle{empty}
\begin{center}
    \vspace*{2.3cm}
    \textbf{\Large{Fusão de evidências na detecção de bordas em Imagens PolSAR}}\\
    
    \vspace*{1.2cm}
    \Large{Anderson Adaime de Borba}
    
    \vskip 2cm
    \textsc{
    Exame de qualificação apresentado\\[-0.25cm] 
    a\\[-0.25cm]
    Faculadade de Computação e informática\\[-0.25cm]
    da\\[-0.25cm]
    Universidade Presbiteriana Mackenzie\\[-0.25cm]
    para\\[-0.25cm]
    obtenção do título\\[-0.25cm]
    de\\[-0.25cm]
    Doutor em Ciências}
    
    \vskip 1.5cm
    Programa de Pós graduação em Engenharia Elétrica e Computação - PPGEEC\\
    Orientador: Prof. Dr. Mauricio Marengoni\\
    Coorientador: Prof. Dr. Alejandro César Frery Orgambide

   	\vskip 1cm
    \normalsize{Durante o desenvolvimento deste trabalho o autor recebeu auxílio
    financeiro da CAPES}
    
    \vskip 0.5cm
    \normalsize{São Paulo, 29  de outubro de 2018}
\end{center}

% ---------------------------------------------------------------------------- %
% Página de rosto (SÓ PARA A VERSÃO DEPOSITADA - ANTES DA DEFESA)
% Resolução CoPGr 5890 (20/12/2010)
%
% IMPORTANTE:
%   Coloque um '%' em todas as linhas
%   desta página antes de compilar a versão
%   final, corrigida, do trabalho
%
%
%\newpage
%\thispagestyle{empty}
%    \begin{center}
%        \vspace*{2.3 cm}
%        \textbf{\Large{Imagens PolSAR}}\\
%        \vspace*{2 cm}
%    \end{center}
%
%    \vskip 2cm
%
%    \begin{flushright}
%	Esta é a versão original da tese elaborada pelo\\
%	candidato Anderson Adaime de Borba, tal como \\
%	submetida à Comissão Julgadora.
%    \end{flushright}
%
%\pagebreak


% ---------------------------------------------------------------------------- %
% Página de rosto (SÓ PARA A VERSÃO CORRIGIDA - APÓS DEFESA)
% Resolução CoPGr 5890 (20/12/2010)
%
% Nota: O título para as dissertações/teses do IME-USP devem caber em um 
% orifício de 10,7cm de largura x 6,0cm de altura que há na capa fornecida pela SPG.
%
% IMPORTANTE:
%   Coloque um '%' em todas as linhas desta
%   página antes de compilar a versão do trabalho que será entregue
%   à Comissão Julgadora antes da defesa
%
%
%\newpage
%\thispagestyle{empty}
%    \begin{center}
%        \vspace*{2.3 cm}
%        \textbf{\Large{Imagens PolSAR}}\\
%        \vspace*{2 cm}
%    \end{center}
%
%    \vskip 2cm
%
%    \begin{flushright}
%	Esta versão da tese contém as correções e alterações sugeridas\\
%	pela Comissão Julgadora durante a defesa da versão original do trabalho,\\
%	realizada em **/**/****. Uma cópia da versão original está disponível no\\
%	na faculdade de Computação e informática da Universidade Presbiteriana Mackenzie.
%
%    \vskip 2cm
%
%    \end{flushright}
%    \vskip 4.2cm
%
%    \begin{quote}
%    \noindent Comissão Julgadora:
%    
%    \begin{itemize}
%		\item Profª. Drª. Nome Completo (orientadora) - Mackenzie [sem ponto final]
%		\item Prof. Dr. Nome Completo - UFAL [sem ponto final]
%		\item Prof. Dr. Nome Completo - UFAL [sem ponto final]
%    \end{itemize}
%      
%    \end{quote}
%\pagebreak
%
%
%\pagenumbering{roman}     % começamos a numerar 

% ---------------------------------------------------------------------------- %
% Agradecimentos:
% Se o candidato não quer fazer agradecimentos, deve simplesmente eliminar esta página 
%\chapter*{Agradecimentos}
%Texto texto texto texto texto texto texto texto texto texto texto texto texto
%texto texto texto texto texto texto texto texto texto texto texto texto texto
%texto texto texto texto texto texto texto texto texto texto texto texto texto
%texto texto texto texto. Texto opcional.


% ---------------------------------------------------------------------------- %
% Resumo
%\chapter*{Resumo}

%\noindent BORBA, A. A. \textbf{Imagens PolSAR}. 
%2019. *** f.
%Tese (Doutorado) - Faculadade de Computação e Informática,
%Universidade Presbiteriana Mackenzie, São Paulo, 2019.
%\\
%
%Elemento obrigatório, constituído de uma sequência de frases concisas e
%objetivas, em forma de texto.  Deve apresentar os objetivos, métodos empregados,
%resultados e conclusões.  O resumo deve ser redigido em parágrafo único, conter
%no máximo 500 palavras e ser seguido dos termos representativos do conteúdo do
%trabalho (palavras-chave). 
%Texto texto texto texto texto texto texto texto texto texto texto texto texto
%texto texto texto texto texto texto texto texto texto texto texto texto texto
%texto texto texto texto texto texto texto texto texto texto texto texto texto
%texto texto texto texto texto texto texto texto texto texto texto texto texto
%texto texto texto texto texto texto texto texto texto texto texto texto texto
%texto texto texto texto texto texto texto texto.
%Texto texto texto texto texto texto texto texto texto texto texto texto texto
%texto texto texto texto texto texto texto texto texto texto texto texto texto
%texto texto texto texto texto texto texto texto texto texto texto texto texto
%texto texto texto texto texto texto texto texto texto texto texto texto texto
%texto texto.
%\\
%
%\noindent \textbf{Palavras-chave:} palavra-chave1, palavra-chave2, palavra-chave3.

% ---------------------------------------------------------------------------- %
% Abstract
%\chapter*{Abstract}
%\noindent BORBA, A. A. \textbf{Imagens PolSAR}. 
%2019. *** f.
%Tese (Doutorado) - Faculadade de Computação e Informática,
%Universidade Presbiteriana Mackenzie, São Paulo, 2019.
%\\


%Elemento obrigatório, elaborado com as mesmas características do resumo em
%língua portuguesa. De acordo com o Regimento da Pós- Graduação, deve ser redigido em inglês para fins de divulgação. 
%Text text text text text text text text text text text text text text text text
%text text text text text text text text text text text text text text text text
%text text text text text text text text text text text text text text text text
%text text text text text text text text text text text text.
%Text text text text text text text text text text text text text text text text
%text text text text text text text text text text text text text text text text
%text text text.
%\\
%
%\noindent \textbf{Keywords:} keyword1, keyword2, keyword3.

% ---------------------------------------------------------------------------- %
% Sumário
\tableofcontents    % imprime o sumário

% ---------------------------------------------------------------------------- %
\chapter{Lista de Abreviaturas}
\begin{tabular}{ll}
SAR            & Imagens obtidas com radar de abertura sintética\\
PolSAR         & Imagens obtidas com radar de abertura sintética polarimética\\
\textbf{p.d.f} & Função densidade de probabilidade\\
LoG            & Detector de borda usando o laplaciano da gaussiana\\
\textbf{MMQ}   & Método dos quadrados mínimos\\
\textbf{MLE}   & Método de estimativa de máxima verossimilhança\\
\textbf{GenSA} & Método Simullated anneling\\
\textbf{ROI} & Método Simullated anneling\\
\end{tabular}

% ---------------------------------------------------------------------------- %
\chapter{Lista de Símbolos}
\begin{tabular}{ll}
        $L$         & Número de visadas em uma imagem PolSAR\\
        $\Sigma$    & Matriz de covariância hermitiana e definida positiva \\
	$E[\cdot]$  & Valor esperado\\
	$\Gamma$    & Função Gamma \\
	$\Gamma_m$  & Função Gamma multivariada\\
	$W(\Gamma, L)$ & Distribuíção Wishart\\
\end{tabular}

% ---------------------------------------------------------------------------- %
% Listas de figuras e tabelas criadas automaticamente
\listoffigures            
\listoftables            

% ---------------------------------------------------------------------------- %
% Capítulos do trabalho
\mainmatter

% cabeçalho para as páginas de todos os capítulos
\fancyhead[RE,LO]{\thesection}

\singlespacing              % espaçamento simples
%\onehalfspacing            % espaçamento um e meio

%\input cap-introducao        % associado ao arquivo: 'cap-introducao.tex'
%\input cap-conceitos         % associado ao arquivo: 'cap-conceitos.tex'
\input cap2acf               % AAB - Arquivo inserido
\input cap3fusao             % AAB - Arquivo inserido
\input cap4metrica           % AAB - Arquivo inserido
%\input cap-conclusoes        % associado ao arquivo: 'cap-conclusoes.tex'
%\input apendice

% cabeçalho para os apêndices
\renewcommand{\chaptermark}[1]{\markboth{\MakeUppercase{\appendixname\ \thechapter}} {\MakeUppercase{#1}} }
\fancyhead[RE,LO]{}
\appendix

%\chapter{Sequ�ncias}
\label{ape:sequencias}

Texto texto texto texto texto texto texto texto texto texto texto texto texto
texto texto texto texto texto texto texto texto texto texto texto texto texto
texto texto texto texto texto texto.


\singlespacing

\renewcommand{\arraystretch}{0.85}
\captionsetup{margin=1.0cm}  % corre��o nas margens dos captions.
%--------------------------------------------------------------------------------------
\begin{table}
\begin{center}
\begin{small}
\begin{tabular}{|c|c|c|c|c|c|c|c|c|c|c|c|c|} 
\hline
\emph{Limiar} & 
\multicolumn{3}{c|}{MGWT} & 
\multicolumn{3}{c|}{AMI} &  
\multicolumn{3}{c|}{\emph{Spectrum} de Fourier} & 
\multicolumn{3}{c|}{Caracter�sticas espectrais} \\
\cline{2-4} \cline{5-7} \cline{8-10} \cline{11-13} & 
\emph{Sn} & \emph{Sp} & \emph{AC} & 
\emph{Sn} & \emph{Sp} & \emph{AC} & 
\emph{Sn} & \emph{Sp} & \emph{AC} & 
\emph{Sn} & \emph{Sp} & \emph{AC}\\ \hline \hline
 1 & 1.00 & 0.16 & 0.08 & 1.00 & 0.16 & 0.08 & 1.00 & 0.16 & 0.08 & 1.00 & 0.16 & 0.08 \\
 2 & 1.00 & 0.16 & 0.09 & 1.00 & 0.16 & 0.09 & 1.00 & 0.16 & 0.09 & 1.00 & 0.16 & 0.09 \\
 2 & 1.00 & 0.16 & 0.10 & 1.00 & 0.16 & 0.10 & 1.00 & 0.16 & 0.10 & 1.00 & 0.16 & 0.10 \\
 4 & 1.00 & 0.16 & 0.10 & 1.00 & 0.16 & 0.10 & 1.00 & 0.16 & 0.10 & 1.00 & 0.16 & 0.10 \\
 5 & 1.00 & 0.16 & 0.11 & 1.00 & 0.16 & 0.11 & 1.00 & 0.16 & 0.11 & 1.00 & 0.16 & 0.11 \\
 6 & 1.00 & 0.16 & 0.12 & 1.00 & 0.16 & 0.12 & 1.00 & 0.16 & 0.12 & 1.00 & 0.16 & 0.12 \\
 7 & 1.00 & 0.17 & 0.12 & 1.00 & 0.17 & 0.12 & 1.00 & 0.17 & 0.12 & 1.00 & 0.17 & 0.13 \\
 8 & 1.00 & 0.17 & 0.13 & 1.00 & 0.17 & 0.13 & 1.00 & 0.17 & 0.13 & 1.00 & 0.17 & 0.13 \\
 9 & 1.00 & 0.17 & 0.14 & 1.00 & 0.17 & 0.14 & 1.00 & 0.17 & 0.14 & 1.00 & 0.17 & 0.14 \\
10 & 1.00 & 0.17 & 0.15 & 1.00 & 0.17 & 0.15 & 1.00 & 0.17 & 0.15 & 1.00 & 0.17 & 0.15 \\
11 & 1.00 & 0.17 & 0.15 & 1.00 & 0.17 & 0.15 & 1.00 & 0.17 & 0.15 & 1.00 & 0.17 & 0.15 \\
12 & 1.00 & 0.18 & 0.16 & 1.00 & 0.18 & 0.16 & 1.00 & 0.18 & 0.16 & 1.00 & 0.18 & 0.16 \\
13 & 1.00 & 0.18 & 0.17 & 1.00 & 0.18 & 0.17 & 1.00 & 0.18 & 0.17 & 1.00 & 0.18 & 0.17 \\
14 & 1.00 & 0.18 & 0.17 & 1.00 & 0.18 & 0.17 & 1.00 & 0.18 & 0.17 & 1.00 & 0.18 & 0.17 \\
15 & 1.00 & 0.18 & 0.18 & 1.00 & 0.18 & 0.18 & 1.00 & 0.18 & 0.18 & 1.00 & 0.18 & 0.18 \\
16 & 1.00 & 0.18 & 0.19 & 1.00 & 0.18 & 0.19 & 1.00 & 0.18 & 0.19 & 1.00 & 0.18 & 0.19 \\
17 & 1.00 & 0.19 & 0.19 & 1.00 & 0.19 & 0.19 & 1.00 & 0.19 & 0.19 & 1.00 & 0.19 & 0.19 \\
17 & 1.00 & 0.19 & 0.20 & 1.00 & 0.19 & 0.20 & 1.00 & 0.19 & 0.20 & 1.00 & 0.19 & 0.20 \\
19 & 1.00 & 0.19 & 0.21 & 1.00 & 0.19 & 0.21 & 1.00 & 0.19 & 0.21 & 1.00 & 0.19 & 0.21 \\
20 & 1.00 & 0.19 & 0.22 & 1.00 & 0.19 & 0.22 & 1.00 & 0.19 & 0.22 & 1.00 & 0.19 & 0.22 \\ \hline 
\end{tabular}
\caption{Exemplo de tabela.}
\label{tab:tab:F5}
\end{small}
\end{center}
\end{table}

      % associado ao arquivo: 'ape-conjuntos.tex'

% ---------------------------------------------------------------------------- %
% Bibliografia
\backmatter \singlespacing   % espaçamento simples
\bibliographystyle{plainnat-ime} % citação bibliográfica textual
\bibliography{../bibliografia}  % associado ao arquivo: 'bibliografia.bib'
% ---------------------------------------------------------------------------- %
% Índice remissivo
\index{TBP|see{periodicidade região codificante}}
\index{DSP|see{processamento digital de sinais}}
\index{STFT|see{transformada de Fourier de tempo reduzido}}
\index{DFT|see{transformada discreta de Fourier}}
\index{Fourier!transformada|see{transformada de Fourier}}

\printindex   % imprime o índice remissivo no documento 

\end{document}
